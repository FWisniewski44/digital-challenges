% Options for packages loaded elsewhere
\PassOptionsToPackage{unicode}{hyperref}
\PassOptionsToPackage{hyphens}{url}
\PassOptionsToPackage{dvipsnames,svgnames,x11names}{xcolor}
%
\documentclass[
  12pt,
  english,
]{article}
\title{\includegraphics[width=4cm,height=\textheight]{/Users/flo/Documents/backup_doc_linux/uni-bamberg.png}

What to make of citizens' attitudes towards technocracy?}
\usepackage{etoolbox}
\makeatletter
\providecommand{\subtitle}[1]{% add subtitle to \maketitle
  \apptocmd{\@title}{\par {\large #1 \par}}{}{}
}
\makeatother
\subtitle{A quantitative exploratory study examining a way ahead in
technocratic attitudes research}
\author{Florian Wisniewski\footnote{Otto-Friedrich-Universität Bamberg
  (Matriculation Nr.: 2028075) - 3rd semester, M. A. Political Science} \and Am
Werkkanal 9, 96047 Bamberg \textbar{}
\href{mailto:florian.wisniewski@stud.uni-bamberg.de}{\nolinkurl{florian.wisniewski@stud.uni-bamberg.de}} \and Seminar:
PWM-CS-HS2: Digital Challenges to Institutions \and Lecturer:
Prof.~Dr.~Andreas Jungherr}
\date{15/04/2022}

\usepackage{amsmath,amssymb}
\usepackage{lmodern}
\usepackage{setspace}
\usepackage{iftex}
\ifPDFTeX
  \usepackage[T1]{fontenc}
  \usepackage[utf8]{inputenc}
  \usepackage{textcomp} % provide euro and other symbols
\else % if luatex or xetex
  \usepackage{unicode-math}
  \defaultfontfeatures{Scale=MatchLowercase}
  \defaultfontfeatures[\rmfamily]{Ligatures=TeX,Scale=1}
  \setmainfont[]{Times New Roman}
\fi
% Use upquote if available, for straight quotes in verbatim environments
\IfFileExists{upquote.sty}{\usepackage{upquote}}{}
\IfFileExists{microtype.sty}{% use microtype if available
  \usepackage[]{microtype}
  \UseMicrotypeSet[protrusion]{basicmath} % disable protrusion for tt fonts
}{}
\makeatletter
\@ifundefined{KOMAClassName}{% if non-KOMA class
  \IfFileExists{parskip.sty}{%
    \usepackage{parskip}
  }{% else
    \setlength{\parindent}{0pt}
    \setlength{\parskip}{6pt plus 2pt minus 1pt}}
}{% if KOMA class
  \KOMAoptions{parskip=half}}
\makeatother
\usepackage{xcolor}
\IfFileExists{xurl.sty}{\usepackage{xurl}}{} % add URL line breaks if available
\IfFileExists{bookmark.sty}{\usepackage{bookmark}}{\usepackage{hyperref}}
\hypersetup{
  pdfauthor={Florian Wisniewski; Am Werkkanal 9, 96047 Bamberg \textbar{} florian.wisniewski@stud.uni-bamberg.de; Seminar: PWM-CS-HS2: Digital Challenges to Institutions; Lecturer: Prof.~Dr.~Andreas Jungherr},
  pdflang={EN},
  pdfkeywords={technocracy --- attitudinal research --- expert
government --- citizen-centered --- quantitative study},
  colorlinks=true,
  linkcolor={blue},
  filecolor={Maroon},
  citecolor={blue},
  urlcolor={blue},
  pdfcreator={LaTeX via pandoc}}
\urlstyle{same} % disable monospaced font for URLs
\usepackage[left=2.5cm,right=2.5cm,top=2cm,bottom=2cm]{geometry}
\usepackage{longtable,booktabs,array}
\usepackage{calc} % for calculating minipage widths
% Correct order of tables after \paragraph or \subparagraph
\usepackage{etoolbox}
\makeatletter
\patchcmd\longtable{\par}{\if@noskipsec\mbox{}\fi\par}{}{}
\makeatother
% Allow footnotes in longtable head/foot
\IfFileExists{footnotehyper.sty}{\usepackage{footnotehyper}}{\usepackage{footnote}}
\makesavenoteenv{longtable}
\usepackage{graphicx}
\makeatletter
\def\maxwidth{\ifdim\Gin@nat@width>\linewidth\linewidth\else\Gin@nat@width\fi}
\def\maxheight{\ifdim\Gin@nat@height>\textheight\textheight\else\Gin@nat@height\fi}
\makeatother
% Scale images if necessary, so that they will not overflow the page
% margins by default, and it is still possible to overwrite the defaults
% using explicit options in \includegraphics[width, height, ...]{}
\setkeys{Gin}{width=\maxwidth,height=\maxheight,keepaspectratio}
% Set default figure placement to htbp
\makeatletter
\def\fps@figure{htbp}
\makeatother
\setlength{\emergencystretch}{3em} % prevent overfull lines
\providecommand{\tightlist}{%
  \setlength{\itemsep}{0pt}\setlength{\parskip}{0pt}}
\setcounter{secnumdepth}{5}
\newlength{\cslhangindent}
\setlength{\cslhangindent}{1.5em}
\newlength{\csllabelwidth}
\setlength{\csllabelwidth}{3em}
\newlength{\cslentryspacingunit} % times entry-spacing
\setlength{\cslentryspacingunit}{\parskip}
\newenvironment{CSLReferences}[2] % #1 hanging-ident, #2 entry spacing
 {% don't indent paragraphs
  \setlength{\parindent}{0pt}
  % turn on hanging indent if param 1 is 1
  \ifodd #1
  \let\oldpar\par
  \def\par{\hangindent=\cslhangindent\oldpar}
  \fi
  % set entry spacing
  \setlength{\parskip}{#2\cslentryspacingunit}
 }%
 {}
\usepackage{calc}
\newcommand{\CSLBlock}[1]{#1\hfill\break}
\newcommand{\CSLLeftMargin}[1]{\parbox[t]{\csllabelwidth}{#1}}
\newcommand{\CSLRightInline}[1]{\parbox[t]{\linewidth - \csllabelwidth}{#1}\break}
\newcommand{\CSLIndent}[1]{\hspace{\cslhangindent}#1}
\usepackage{booktabs}
\usepackage{longtable}
\usepackage{array}
\usepackage{multirow}
\usepackage{wrapfig}
\usepackage{float}
\usepackage{colortbl}
\usepackage{pdflscape}
\usepackage{tabu}
\usepackage{threeparttable}
\usepackage{threeparttablex}
\usepackage[normalem]{ulem}
\usepackage{makecell}
\usepackage{xcolor}
\ifXeTeX
  % Load polyglossia as late as possible: uses bidi with RTL langages (e.g. Hebrew, Arabic)
  \usepackage{polyglossia}
  \setmainlanguage[]{english}
\else
  \usepackage[main=english]{babel}
% get rid of language-specific shorthands (see #6817):
\let\LanguageShortHands\languageshorthands
\def\languageshorthands#1{}
\fi
\ifLuaTeX
  \usepackage{selnolig}  % disable illegal ligatures
\fi

\begin{document}
\maketitle
\begin{abstract}
In political science research, research on technocracy still is an
evolving topic. Compared to other ones, e. g. populism, there have not
been as many explorative studies in this direction. Whilst technocracy
has been examined in more detail on state level, a citizen-centered
approach is still scarce, but in the making. This paper will add to this
part of the literature. Building on core hypotheses of existing studies,
it examines already tested hypotheses with newer data and in five
selected countries, finding both proof in favour of current scholarly
beliefs, but also contesting evidence in some regard. Furthermore, a new
hypothesis about citizens' attitudes towards science and the
relationship of it with technocratic attitudes is put forth. The paper
concludes with remarks on further scholarly endeavours and ways to
enhance attitudinal research in the research space of technocracy and
expert government.
\end{abstract}

\setstretch{1.5}
\newpage{}

\tableofcontents

\newpage{}

\hypertarget{introduction}{%
\section{Introduction}\label{introduction}}

One of the most discussed themes in political science over the last
years most definitely was populism. Especially in the US -- with the
rise of Donald Trump to become the 45th president in 2017 and the
accompanying transformation of the Republican Party --- this phenomenon
evolved into a well-researched topic. But not only there: also in
Europe, the rise of populism and populist politicians became a big
issue, also within the scientific community --- namely triggered by e.
g. left-wing populist SYRIZA's rise to power in Greece, the emergence of
the AfD as a right-wing populist party in Germany, or the presidential
race in France being decided between the liberal candidate Macron and
the right-wing or even far right populist Marine Le Pen just as of the
writing of this paper in April 2022. When talking and writing about it,
one comes across many different definitions to the same term populism:
Cas Mudde's `populism as a thin ideology'
(\protect\hyperlink{ref-mudde2004populist}{Mudde, 2004}), Ernesto
Laclau's `populism as a discursive strategy'
(\protect\hyperlink{ref-laclau2005populist}{Laclau, 2005}), Jan-Werner
Müller's `methods of populism'
(\protect\hyperlink{ref-muxfcller2016populismus}{Müller, 2016}) or the
`ideational approach' by Kirk Hawkins and Cristóbal Rovira Kaltwasser
(\protect\hyperlink{ref-hawkins2018ideational}{Hawkins \& Rovira
Kaltwasser, 2018}).

But populism is by far not the only `alternative approach' to democracy
in its various forms. One of the other possible views is technocracy,
and research within this space is still very fluid and evolving in this
field within political science. A lot of said research has so far often
concentrated on the more normative aspects of technocracy
(\protect\hyperlink{ref-habermas2015lure}{Habermas, 2015}) or has taken
a rather expert-centered stance, discussing the determination of who the
experts would be and who chooses when someone is considered one
(\protect\hyperlink{ref-bickerton2020technocracy}{Bickerton \& Accetti,
2020}). But there certainly is more to \emph{`technocracy' as a whole
topic}. For example, one could also ask how citizens themselves see this
concept in turn.

That is the goal of this paper: the question that will be addressed is
about citizens' attitudes towards this alternative concept of
government, and how it is influenced by certain political aspects,
values, views, and also demographic variables. Other social scientists
have only very recently taken up this narrative, showing that there is
significant homogeneity throughout people, especiall in different
countries or continents when examining their opinions of technocracy
(see e. g. \protect\hyperlink{ref-bertsou2017technocratic}{Bertsou \&
Pastorella} (\protect\hyperlink{ref-bertsou2017technocratic}{2017}),
\protect\hyperlink{ref-ganuza2020experts}{Ganuza \& Font}
(\protect\hyperlink{ref-ganuza2020experts}{2020}),
\protect\hyperlink{ref-lavezzolo2021will}{Lavezzolo et al.}
(\protect\hyperlink{ref-lavezzolo2021will}{2021}),
\protect\hyperlink{ref-chiru2022wants}{Chiru \& Enyedi}
(\protect\hyperlink{ref-chiru2022wants}{2022}) or
\protect\hyperlink{ref-bertsou2022people}{Bertsou \& Caramani}
(\protect\hyperlink{ref-bertsou2022people}{2022})).

Therefore, taking very recent survey data about citizens' views of
technocracy in five different countries, this exploratory analysis aims
at shedding more light upon the real-world implications of technocracy
as a normative concept with updated data and see if previously obtained
results and hypotheses still hold. Also, as a secondary goal, a new
hypothesis about citizens' views of science and technology shall be
introduced. By the end of the study, new tracks for future research in
this evolving and still young research area shall also be given, thereby
also functioning a bit as a small research note.

\newpage{}

\hypertarget{theoretical-considerations}{%
\section{Theoretical considerations}\label{theoretical-considerations}}

\hypertarget{conceptual-and-normative-background-the-roots-of-technocracy-as-a-concept-and-introductory-definitions}{%
\subsection{Conceptual and normative background --- the `roots' of
technocracy as a concept and introductory
definitions}\label{conceptual-and-normative-background-the-roots-of-technocracy-as-a-concept-and-introductory-definitions}}

As mentioned in the introductory paragraph, technocracy could be
considered a still evolving field of action within the social sciences.
But that does not mean that there are no conclusions to be drawn from
theory. Technocracy as a concept is first mentioned or described by
Plato (stemming from the Greek \emph{`techne'}, translated as `art' or
`craft'). As explained by Bickerton and Accetti, this is what can be
considered the classical argument for technocracy: Plato believed that
to really fulfill the goal of \emph{good} government, i. e. to justly
rule and bring order to the social lives of the people within society,
the one who governs needs to have suitable skills to do so. Those could
be found in the philosophers: they would know, and have those skills
(see \protect\hyperlink{ref-bickerton2020technocracy}{Bickerton \&
Accetti, 2020, p. 32f}.).

Another aspect that describes the difference between (representatively)
democratic and technocratic systems has been given by Robert A. Dahl in
his classic `On Democracy' in the chapter `Ideal Democracy.' He takes a
position of staunch resistance toward technocratic expert government,
stating: \emph{`Because experts may be qualified to serve as your agents
does not mean that they are qualified to serve as your rulers'}
(\protect\hyperlink{ref-dahl_onDemocracy}{Dahl, 2020, p. 73}).

Therewith, he made the exact opposite point to Plato's argument
described above. The latter assumed the philosopher-kings (so to say,
the `proto-technocrats' of Ancient Greece) to be the best possible
rulers because they would possess a certain set of skills that the broad
masses did not have and would apply it to achieve the greater good for
all of society --- democracy as a form of governance, in turn, could
cause chaos. Dahl on the other hand brings forth the notion of
\emph{corruption through power}, by which their proclaimed pursuit of
the `greater good' for the whole society can be undermined, which he
sees proven by history, too (see
\protect\hyperlink{ref-dahl_onDemocracy}{Dahl, 2020, p. 74}). What the
advocates of expertise tried to do, is, according to Dahl, attacking
democracy at a perceived weakness: ordinary people, so they claim, would
lack the competence to govern --- with the Guardians having it by their
superior knowledge of the greater good and the means to achieve it for
the society. Dahl also points to another important aspect of this
debate: the lack of competence of the citizens does not, in the
advocates for expert rule's eyes, diminish the fact that all people or
their interests were equal. According to him their perceived superior
\emph{knowledge} is the reason why the latter claim to be better suited
to rule --- so that everyone could be able to profit from it (see
\protect\hyperlink{ref-dahl_onDemocracy}{Dahl, 2020, pp. 69--71}).

Another last argument made by Dahl shall be laid out here: the critique
of a general scientific way of decision-making in politics. He points to
imporant policy-related questions such as state budget creation and
assignment or welfare state policies. Those are, to him, areas that
present an unmountable challenge to this expert-centered, scientific
decision-making process, as he sees no possibility of the existence of a
group of people who could satisfactorily provide an answer to these
questions (see \protect\hyperlink{ref-dahl_onDemocracy}{Dahl, 2020, p.
72f}.).

To further shed light on the term `technocracy' itself, another
definition given by
\protect\hyperlink{ref-caramani2020technocratic}{Caramani}
(\protect\hyperlink{ref-caramani2020technocratic}{2020}) should be
considered. He considers it to be `a form of power in which decisions
over the allocation of values are made by experts or technical elites
based on their knowledge, independently and in the long-term interest of
the whole of society'
(\protect\hyperlink{ref-caramani2020technocratic}{Caramani, 2020, p.
3}), further adding four features making government technocratic in
nature (see \protect\hyperlink{ref-caramani2020technocratic}{Caramani,
2020, p. 3f}.):

\begin{enumerate}
\def\labelenumi{\arabic{enumi}.}
\item
  The source of power is knowledge, rather than popular will (e. g.
  through elections).
\item
  Political representation is realized via a „trustee model``, where
  citizens lack the knowledge and wisdom to enter the political arena
  themselves. Their „trustees`` (i. e. the experts) do that for them,
  turning democracy and elections into a mere elite selection process.
\item
  Politics is seen by the expert elites as a means to enhance the
  „greater good`` and acting in the common interest.
\item
  The common interest itself is assessed through science, rationally,
  and as objective as possible, not by ideology or derived from
  pluralist preference aggregation within the people.
\end{enumerate}

But for this paper, it is also important to discuss how technocracy may
be measured. When doing attitudinal research with citizens, especially
concerning elusive and heterogeneous topics like technocracy,
researchers have to be careful --- as not every respondent understands
the same under such complex concepts. This is why one should also take a
look at what empirical studies of the concept, both with focus on a
state- and a citizen-level have already found so far, and how they did
it.

\newpage{}

\hypertarget{the-way-so-far-empirical-and-theoretical-findings-on-technocracy-and-technocratic-attitudes}{%
\subsection{`The way so far': empirical and theoretical findings on
technocracy and technocratic
attitudes}\label{the-way-so-far-empirical-and-theoretical-findings-on-technocracy-and-technocratic-attitudes}}

Now, after having defined in detail the implications of technocratic
systems, the area of tension between representative democracy and
technocracy, and the problem this paper wants to address, it is also
necessary, as mentioned before, to take a look at what empirical studies
have found out so far.

To start off, \protect\hyperlink{ref-mcdonnell2014defining}{McDonnell \&
Valbruzzi} (\protect\hyperlink{ref-mcdonnell2014defining}{2014}) point
out an important discovery, also to an extent relevant in the context of
this paper: there is a recognizably different thrust throughout Europe
when talking about the degree to which technocracy may be witnessed
within national governments. They set three basic rules as to when a
government is to be regarded as an ideal type technocratic one (see
\protect\hyperlink{ref-mcdonnell2014defining}{McDonnell \& Valbruzzi,
2014, p. 656}):

\begin{enumerate}
\def\labelenumi{\arabic{enumi}.}
\tightlist
\item
  Elected officials of parties do \emph{not} make all governmental
  decisions.
\item
  Policies do not have their roots within parties which after deciding
  them, act to implement them.
\item
  High officials such as (prime) ministers do not have a party
  background.
\end{enumerate}

By applying these rules, they classify governments in Europe between
1945 and 2013. They also discern between full technocratic governments
and technocrat-led party governments. Examining their results, they see
two implications: technocratic governments form more easily in countries
where there is a stronger head of state, and this government type is
more frequently observed in countries with weaker party systems or
those, that only recently developed a democratic political system (see
\protect\hyperlink{ref-mcdonnell2014defining}{McDonnell \& Valbruzzi,
2014, p. 666}).

\protect\hyperlink{ref-font2015participation}{Font et al.}
(\protect\hyperlink{ref-font2015participation}{2015}) apply these
findings in their study about Spaniards attitudes towards political
participation and political decision-making. While the results of
\protect\hyperlink{ref-mcdonnell2014defining}{McDonnell \& Valbruzzi}
(\protect\hyperlink{ref-mcdonnell2014defining}{2014}) were centered at
the state-level, they went for a different approach, concentrating more
on the `demand side' for technocracy and expert government. They find,
that expert-based decision-making is indeed viewed as a third
alternative by citizens in Spain (with the other two being democratic
ones: representative and participative decision-making). Explaining this
distinction and also the citizens' tendency to actually notice
expert-based government, they point to several factors, e. g. lacking
trust in either elected representatives or their citizen peers as far as
political decisions are concerned, a desire to incorporate efficiency
and expertise into the decision process, and/or wanting technocratic
actors to play a more central role in said process. Concluding their
study, they state that for some citizens, the contrast to their
currently in-place political system is very important (like e. g. in
Spain, representative democracy), so that those with more favourable
views for expert government could be supposed to be more dissatisfied
with it --- and therefore rendering \emph{any} alternative more
appealing (see \protect\hyperlink{ref-font2015participation}{Font et
al., 2015, p. 165f}.).

Another central finding in the technocracy and expert government
research sphere was put forth by
\protect\hyperlink{ref-lavezzolo2021will}{Lavezzolo et al.}
(\protect\hyperlink{ref-lavezzolo2021will}{2021}). They mention an
important distinction between what technocracy is and in which forms it
may occur. Their explanations and results are also important for this
paper, because it sheds more light on what people may actually think and
most importantly \emph{understand} when asked about `technocracy.'
Especially important is their description of so-called \emph{technopols}
as the form technocracy takes most often in everyday politics.
Technopols are full party politicians that nevertheless exert expertise
to a certain degree, having a technical/knowledge-based background. The
authors also point out that there is a strong demand for this type of
politician, (see \protect\hyperlink{ref-lavezzolo2021will}{Lavezzolo et
al., 2021, p. 1507}; for the concept of technopols examined in more
detail, see \protect\hyperlink{ref-joignant2011technopols}{Joignant,
2011})

\hypertarget{hypotheses}{%
\subsection{Hypotheses}\label{hypotheses}}

Having explored both the theoretical background of technocracy as a
concept under 2.1 and some empirical findings under 2.2, the hypotheses
that shall be tested in this paper will now be stated. Those will
consist of the `tried-and-tested ones' in this research space, taken
also from the aforementioned studies under 2.2 --- but tested with newer
data (H1-H6). Also, to supplement scholarly work, a new hypothesis will
be generated and theoretically justified in the course of this paragraph
(H7).

\hypertarget{hypotheses-on-the-way-so-far}{%
\subsubsection{Hypotheses on `the way so
far'}\label{hypotheses-on-the-way-so-far}}

The first hypothesis, also having been constantly checked and updated by
scholars with newer data becoming available over the years, concerns
citizens' attitudes towards democracy as a form of governance (see e. g.
\protect\hyperlink{ref-bengtsson2009direct}{Bengtsson \& Mattila}
(\protect\hyperlink{ref-bengtsson2009direct}{2009}),
\protect\hyperlink{ref-coffe2014education}{Coffé \& Michels}
(\protect\hyperlink{ref-coffe2014education}{2014}) and
\protect\hyperlink{ref-bertsou2017technocratic}{Bertsou \& Pastorella}
(\protect\hyperlink{ref-bertsou2017technocratic}{2017})). Those studies
originate from different directions, with some measuring citizens'
support for stealth democracy, others for technocracy and expert
government. Also, they yielded different results, as
\protect\hyperlink{ref-chiru2022wants}{Chiru \& Enyedi}
(\protect\hyperlink{ref-chiru2022wants}{2022}) recently pointed out.
This is why the hypothesis shall be tested again in the context of this
paper. The reasoning behind this follows
\protect\hyperlink{ref-bertsou2017technocratic}{Bertsou \& Pastorella}
(\protect\hyperlink{ref-bertsou2017technocratic}{2017}): citizens that
generally trust democratic institutions and players are satisfied with
them and the system as a whole will not be fond of technocracy as a
sort-of `contender' to it. Therefore, the first hypothesis will be:

\begin{quote}
H1: The more satisfied people are with democracy as a political system,
the less likely it becomes for them to support technocracy/expert
government.
\end{quote}

The next hypothesis is grounded in the contrast of technocracy and
democracy regarding institutions. As put forth by
\protect\hyperlink{ref-bertsou2017technocratic}{Bertsou \& Pastorella}
(\protect\hyperlink{ref-bertsou2017technocratic}{2017}), if citizens
have had experiences that would diminish their trust in democracy's
representative institutions (as in e. g. a series of unreliable
governments that made ineffective decisions when implementing policies),
this could lead to them being more receptive for the ideas of experts
governing --- believing in the notion that they could do it in a better
way. On the contrary, citizens that have more trust in their political
system of representative democracy would be less likely to have a
positive opinion on technocracy (see
\protect\hyperlink{ref-bertsou2017technocratic}{Bertsou \& Pastorella,
2017, p. 434f}.). Another important criticism of democratic
decision-making by technocratically minded citizens is highlighted by
\protect\hyperlink{ref-bertsou2022people}{Bertsou \& Caramani}
(\protect\hyperlink{ref-bertsou2022people}{2022}), where it is stated
that the lower trust of those people comes from their perceived
ineffectivity and irresponsibility with democratic decision-making and
-makers. This paper therefore tests the following as a second
hypothesis:

\begin{quote}
H2: Technocracy and expert government are seen more negatively when
trust in political institutions of representative democracy is higher.
\end{quote}

In a similar vein,
\protect\hyperlink{ref-bertsou2017technocratic}{Bertsou \& Pastorella}
(\protect\hyperlink{ref-bertsou2017technocratic}{2017}) have found
evidence that people with higher political interest would see
technocratic government in a more negative view. They derived this from
the stealth democracy literature (see e. g.
\protect\hyperlink{ref-hibbing2002stealth}{Hibbing \& Theiss-Morse}
(\protect\hyperlink{ref-hibbing2002stealth}{2002}) and
\protect\hyperlink{ref-coffe2014education}{Coffé \& Michels}
(\protect\hyperlink{ref-coffe2014education}{2014})) (see
\protect\hyperlink{ref-bertsou2017technocratic}{Bertsou \& Pastorella,
2017, p. 435}), listing the variable in their pool of controls for their
analysis. Therefore, this paper's models will also contain a fitting
variable, checking for the effect with the following hypothesis:

\begin{quote}
H3: Technocracy and expert government are seen more negatively when
citizens' political interest is higher.
\end{quote}

As already pointed out by
\protect\hyperlink{ref-bengtsson2009direct}{Bengtsson \& Mattila}
(\protect\hyperlink{ref-bengtsson2009direct}{2009}) as part of the
research on stealth democracy, it has been found out that more
right-wing oriented people favour more expert decision-making over
democratic forms of government (see
\protect\hyperlink{ref-bengtsson2009direct}{Bengtsson \& Mattila, 2009,
p. 1045}). On the other end of this spatial political spectrum,
\protect\hyperlink{ref-donovan2006popular}{Donovan \& Karp}
(\protect\hyperlink{ref-donovan2006popular}{2006}) found out that with
left-wing oriented citizens, it is quite the contrary: in some
countries, those people were more likely than political `moderates' to
favour the use of popular referendums, thus they would prefer more
direct citizen intervention in politics than right-wing oriented people
(see \protect\hyperlink{ref-donovan2006popular}{Donovan \& Karp, 2006,
pp. 681--684}). Previously, this reasoning has been empirically proven
by e. g. \protect\hyperlink{ref-bertsou2017technocratic}{Bertsou \&
Pastorella} (\protect\hyperlink{ref-bertsou2017technocratic}{2017}) or
\protect\hyperlink{ref-bertsou2022people}{Bertsou \& Caramani}
(\protect\hyperlink{ref-bertsou2022people}{2022}) and also partially by
\protect\hyperlink{ref-chiru2022wants}{Chiru \& Enyedi}
(\protect\hyperlink{ref-chiru2022wants}{2022}). From that background, to
see if this will also hold in the context of this paper and to further
empirically contest the reasoning behind it, the fourth hypothesis reads
as follows:

\begin{quote}
H4: The further on the ideological/political right citizens perceive
themselves to be, the more likely it is for them to support
technocracy/expert government.
\end{quote}

For the fifth hypothesis, this paper turns again to the findings of
\protect\hyperlink{ref-chiru2022wants}{Chiru \& Enyedi}
(\protect\hyperlink{ref-chiru2022wants}{2022}). In their study, they
have taken the state-level hypothesis of
\protect\hyperlink{ref-bertsou2017technocratic}{Bertsou \& Pastorella}
(\protect\hyperlink{ref-bertsou2017technocratic}{2017}), but have
applied it on the citizen-level. They justified this by stating the
existence of other factors that, for corruption perception, had
unintended interaction effects on the country-level, therefore making a
citizen-level inspection necessary. As also shown by e. g.
\protect\hyperlink{ref-chiru2012voter}{Chiru \& Gherghina}
(\protect\hyperlink{ref-chiru2012voter}{2012}) and
\protect\hyperlink{ref-ecker2016corruption}{Ecker et al.}
(\protect\hyperlink{ref-ecker2016corruption}{2016}), incumbent
governments, parties, and politicians may well be held accountable for
their corruption, if perceived by citizens --- yet again, on the other
hand, it has also been discussed that there is a possibility for them to
remain in power regardless, if there is no viable alternative for voters
(see \protect\hyperlink{ref-charron2016ideology}{Charron \& Bågenholm}
(\protect\hyperlink{ref-charron2016ideology}{2016}) or
\protect\hyperlink{ref-hooghe2011distrusting}{Hooghe et al.}
(\protect\hyperlink{ref-hooghe2011distrusting}{2011})). If punished or
not, nevertheless, corruption and citizens perceiving it can contribute
to them losing faith in their current political system
(\protect\hyperlink{ref-ziller2015pure}{Ziller \& Schübel}
(\protect\hyperlink{ref-ziller2015pure}{2015})), which is why this paper
hypothesizes:

\begin{quote}
H5: Technocracy is seen more positively by citizens that perceive
corruption to be a major problem in the political system of their
country.
\end{quote}

The fourth hypothesis will revolve around citizens' educational level.
Starting off from the stealth democracy literature once again, we first
visit \protect\hyperlink{ref-coffe2014education}{Coffé \& Michels}
(\protect\hyperlink{ref-coffe2014education}{2014}) with an important
contribution to this area. Regarding citizens educational background,
they provided two alternative hypotheses: one relying on \emph{cognitive
mobilization theory} (better education meaning more favourable views of
direct citizen involvement in politics through direct democracy) and the
other on \emph{political dissatisfaction theory} (better education
meaning less favourable views of direct citizen involvement through
direct democracy) (see \protect\hyperlink{ref-coffe2014education}{Coffé
\& Michels, 2014, pp. 3--4}). They find evidence in favour of the
\emph{political dissatisfaction theory}-hypothesis (see
\protect\hyperlink{ref-coffe2014education}{Coffé \& Michels, 2014, p.
6}). On the other hand, \protect\hyperlink{ref-chiru2022wants}{Chiru \&
Enyedi} (\protect\hyperlink{ref-chiru2022wants}{2022}) point out that
the findings in this area have already been contested and are somewhat
contradictory when taking a look at their more recent study. They refer
to e. g. \protect\hyperlink{ref-bengtsson2009direct}{Bengtsson \&
Mattila} (\protect\hyperlink{ref-bengtsson2009direct}{2009}) that found
people with higher education tending to evince lower support for both
direct and stealth democracy than those with lower education. Seeing
stealth democracy and technocracy as somewhat related concepts, a
contrary effect was found: people with higher education showed
favourable attitudes towards technocracy more often (see
\protect\hyperlink{ref-chiru2022wants}{Chiru \& Enyedi, 2022, p. 109}).
This is also somewhat supported by
\protect\hyperlink{ref-bertsou2022people}{Bertsou \& Caramani}
(\protect\hyperlink{ref-bertsou2022people}{2022}), as they state:
`technocratic-minded citizens are expected to be highly educated, given
the emphasis placed on the superior skills of a knowledge elite and the
scientific approach to politics'
(\protect\hyperlink{ref-bertsou2022people}{Bertsou \& Caramani, 2022, p.
9}). Therefore, to further shed light on the relation of educational
level with technocratic attitudes and to additionally test the claims
made by \protect\hyperlink{ref-chiru2022wants}{Chiru \& Enyedi}
(\protect\hyperlink{ref-chiru2022wants}{2022}) about people with more
social capital and higher income levels favouring technocratic expert
government, the sixth hypothesis reads as:

\begin{quote}
H6: Citizens that are better educated, have more social capital, and
have more income view technocracy in a more positive way.
\end{quote}

\hypertarget{new-hypothesis-citizens-views-on-science-and-technology}{%
\subsubsection{New hypothesis: citizens' views on science and
technology}\label{new-hypothesis-citizens-views-on-science-and-technology}}

While the first five hypotheses were ones that were already included in
former studies, this paper aims to also contribute a new facet to
attitudinal research concerning technocracy. As previously mentioned,
the findings, especially when talking about the relationship of
education and attitudes towards expert government, are not fully clear;
arguments have been made in favour of what has been found both in
technocracy and in stealth democracy research. Building on suggestions
put forth by \protect\hyperlink{ref-bertsou2020measuring}{Bertsou \&
Caramani} (\protect\hyperlink{ref-bertsou2020measuring}{2020}) for the
measurement of technocracy, it is being deduced that when breaking down
the concept into its parts, one of them is the so-called
\emph{`scientific approach'}. This encompasses that there is some sort
of ground truth and fitting solution to every problem that might arise.
As the authors themselves phrase it:

\begin{quote}
``{[}\ldots{]} Technocratic politics follows a positivist paradigm.
There is a belief in the existence of an optimal solution or truth,
which can be discovered through careful and objective analysis of
scientific evidence. {[}\ldots{]} This scientific approach complements
the aforementioned elitist, anti-political and anti-pluralist dimensions
by emphasizing the role of expert knowledge, political neutrality and
problem-solving capacities.''
(\protect\hyperlink{ref-bertsou2020measuring}{Bertsou \& Caramani, 2020,
p. 5})
\end{quote}

In said article, they also began constructing a basic codebook for
technocratic discourse research. Within this, one dimension deals with
the aforementioned scientific approach, stating that discourse about
technocracy can be viewed as `positive,' if it includes e. g. reference
to objective, neutral, non-compromised decision making, evidence-based
policymaking or critical voices that question facts or the scientific
method. On the contrary, it is viewed as `negative,' if it contains e.
g. doubt of factual evidence, reference to subjective or compromised
decision- and policymakers ordoubt of the complexity of social problems
(see \protect\hyperlink{ref-bertsou2020measuring}{Bertsou \& Caramani,
2020, p. 11}).

Taking into account these theoretical considerations and (preliminary)
findings about the nature of technocracy and technocratic discourse,
some valuable conclusions can be drawn from it about \emph{citizens'}
technocratic attitudes, too. The \emph{`scientific approach'} could also
be converted to the plane of attitudinal research, stating e. g. that
more science-friendly people would also have more technocratic views.
This could be explained by the scientific `mode of thinking': taking e.
g. a positivist paradigm and searching for an optimal solution, being
objective and analytical, reliant on knowledge, and oriented towards
problem solving --- those traits are also imaginable to be found
\emph{within} citizens or to be considered good characteristics
\emph{by} citizens. Such people, when confronted with the choice of some
form of democracy (be it representative or even direct) and technocracy
are expected by this paper to tend to the latter. When e. g. also
considering the implementation process of policies, people that have a
more positive attitude towards science and technology are thought to
rather have experts govern and take care of such policies than `mere'
representative party politicians without expert knowledge or party
expertise. Therefore, H7 takes up these thoughts, stating:

\begin{quote}
H7: People that have a higher opinion of science and technology tend to
view technocracy and government by experts in a more positive light.
\end{quote}

\newpage{}

\hypertarget{data-and-methods}{%
\section{Data and methods}\label{data-and-methods}}

\hypertarget{data-and-caveats}{%
\subsection{Data and caveats}\label{data-and-caveats}}

The data set utilized in this paper is the \emph{World Values Survey}
(wave 7; also called WVS), which features data collected from 2017 until
the end of 2021. As of the writing of this paper, the organization is
still unsure about the end of this wave, making it possible that there
are changes and additions to be made to the data used here. As stated on
the project's website, the WVS is an research programme, studying
people's opinions and attitudes as well as values, highlighting social,
political, economic, religious, and cultural differences and
similarities. It was started by Prof.~Ronald Inglehart back in 1981 and
is, according to their own information, the world's largest
non-commercial academic social survey programme (see
\protect\hyperlink{ref-wvs2022info}{WVS, 2022}). This paper chooses to
work with said data set, as the variable and country availability is
generally very good. Another alternative would have been the European
Values Study (EVS), containing fewer samples only from European states,
and also less variables. The WVS was the preferred option here, as it
also contained information about citizens' attitudes towards science and
technology, which play an important role in this article.

\hypertarget{selection-of-cases}{%
\subsection{Selection of cases}\label{selection-of-cases}}

The countries selected for this paper include Germany, Greece, Romania,
Brazil, and Argentina. Reasons for choosing them within the WVS are
twofold: (1) because of data availability and thereby (2) being able to
follow the argumentation of \protect\hyperlink{ref-chiru2022wants}{Chiru
\& Enyedi} (\protect\hyperlink{ref-chiru2022wants}{2022}). They also
chose those countries (among others), pointing to different experiences
with democratic systems, but also different regime types with e. g.
different backgrounds with technocracy, technocratic decision-making,
and expert government --- therefore providing for a high degree of
variation between the cases, and by that, a good means of testing
hypotheses (see \protect\hyperlink{ref-chiru2022wants}{Chiru \& Enyedi,
2022, p. 100}). Due to data (un-)availability, compared to their study,
the Netherlands, the UK, Switzerland and Hungary had to be excluded,
leaving the countries mentioned above to be researched.

The case selection shall also be explained in a bit more detail per
country, also using typologies and data from
\protect\hyperlink{ref-mcdonnell2014defining}{McDonnell \& Valbruzzi}
(\protect\hyperlink{ref-mcdonnell2014defining}{2014}). Greece and
Romania are both countries that have had significant experience with
technocrat-led or even full technocratic governments. While in Greece
this happened more often more recently (with the Papademos and the
Pikrammenos governments in 2011-2012 and mid-2012 respectively), most of
Romania's experience comes from the 1990s (especially with the Vacaroiu
governments between 1992-1996), but it also encountered one more
recently with the Ciolos cabinet in 2015-2016 (see
\protect\hyperlink{ref-mcdonnell2014defining}{McDonnell \& Valbruzzi}
(\protect\hyperlink{ref-mcdonnell2014defining}{2014}), p.~663, p.~665
and \protect\hyperlink{ref-chiru2022wants}{Chiru \& Enyedi}
(\protect\hyperlink{ref-chiru2022wants}{2022}), p.~103). The German case
represents a country that has not yet experienced such a high degree of
technocracy in its governments, but one could argue that during the
COVID-19 pandemic, a shift to more technocratic or expert-driven
decision-making could be noticed --- as was in many democracies around
the world (see \protect\hyperlink{ref-windholz2020governing}{Windholz}
(\protect\hyperlink{ref-windholz2020governing}{2020}),
\protect\hyperlink{ref-bylund2021separation}{Bylund \& Packard}
(\protect\hyperlink{ref-bylund2021separation}{2021}), or
\protect\hyperlink{ref-kuhlmann2021tracing}{Kuhlmann et al.}
(\protect\hyperlink{ref-kuhlmann2021tracing}{2021})). Also, the media
sometimes pointed out this tendency (see e. g.
\protect\hyperlink{ref-FAZ2020virologen}{Frankfurter Allgemeine Zeitung}
(\protect\hyperlink{ref-FAZ2020virologen}{2020})). Moving on to the two
South American cases, Argentina and Brazil; both of them have
experienced a fair share of non-political or `expert' minister
appointments in the past, with Brazil even more so than Argentina (see
\protect\hyperlink{ref-chiru2022wants}{Chiru \& Enyedi, 2022, p.
102f}.). Furthermore, they provide an additional focus point outside of
Europe, making them valuable additions to the country subset used in
this paper.

\hypertarget{variables-included-and-operationalization-of-concepts-with-the-wvs}{%
\subsection{Variables included and operationalization of concepts with
the
WVS}\label{variables-included-and-operationalization-of-concepts-with-the-wvs}}

Before describing the method used to test the hypotheses of this paper,
it is important to first take a look at the variables and indices used
for testing individually. The dependent variable regarding
\emph{citizens' attitudes towards technocracy} has been recoded for this
academic endeavour to be of binary shape, with 0 indicating a generally
negative view of technocracy and 1 indicating a positive view of the
concept. Occasionally, especially when visualizing the variable in the
univariate part of the paper, its original 4-level structure will be
used, indicating the `mid-steps' of (dis-)agreement with technocracy.

\emph{Self-ascribed position on left-right-scale} is a variable that
measures the position of citizens on a scale of 1 (left) to 10 (right).
The variable was treated as quasi-metric, focusing more on a continuous
depiction of political ideology, rather than seeing each and every level
individually as a `statement' of a citizen. It will be used to check H4.

\emph{Perception of corruption in respondents' respective country} is
used to check H5. It measures on a scale of 1 (no corruption is
perceived in the country) to 10 (abundant corruption is perceived in the
country).

\emph{Political interest} measures the degree to which a citizen is
interested in politics and is being used to test H3. It was recoded into
a binary variable, with 0 meaning ``not politically interested'' and 1
meaning ``politically interested.'' This was done to ease
interpretation.

\emph{Political trust} is an index that was created from three
variables: confidence in national parliament, national government, and
national parties per citizen, just as suggested in
\protect\hyperlink{ref-bertsou2017technocratic}{Bertsou \& Pastorella}
(\protect\hyperlink{ref-bertsou2017technocratic}{2017}). It is used to
check for H2 and ranges from 1 (low trust) to 10 (high trust).
Cronbach's alpha is at 0.87, indicating an excellent fit of the index.

\emph{Satisfaction with democracy} is a variable that has been measured
quite differently throughout technocracy research (see e. g.
\protect\hyperlink{ref-bertsou2017technocratic}{Bertsou \& Pastorella}
(\protect\hyperlink{ref-bertsou2017technocratic}{2017}) vs.
\protect\hyperlink{ref-chiru2022wants}{Chiru \& Enyedi}
(\protect\hyperlink{ref-chiru2022wants}{2022})). In this paper, it is
measured by a single item that reads as follows: ``How satisfied are you
with how the political system is functioning in your country these days?
On the scale from 1 to 10 below, 1 means not satisfied at all and 10
means completely satisfied.'' This is also the reason why results with
this variable may not be fully comparable with other studies but the
variable will be used here to check H1.

\emph{Respondents' age, education, income} are variables that indicate
demographic/economic information about individual citizens. These are
self-explanatory. Educational level (from 1 = low to 3 = high) and
income are part of what is being used to check H6.

\emph{Social capital} is also used in H6. It is an index created for
this study, using
\protect\hyperlink{ref-franzen2007sozialkapital}{Franzen \& Pointner}
(\protect\hyperlink{ref-franzen2007sozialkapital}{2007}). Social capital
is captured in an index formed via membership indicator variables for
church groups, sports clubs, charitable organizations, and
art/music/recreational organizations. Cronbach's alpha for the index is
at 0.71, indicating a fitting selection of items. It ranges from 1 (low
social capital) to 5 (high social capital).

\emph{Attitudes towards science} is the variable used to test the novel
hypothesis H7. It is also an index, ranging from 1 (negative view of
science) to 10 (positive view of science). It was formed from two
fitting items, containing the following (Cronbach's alpha = 0.75):

\begin{itemize}
\tightlist
\item
  ``Science and technology are making our lives healthier, easier, and
  more comfortable''
\item
  ``Because of science and technology, there will be more opportunities
  for the next generation''
\end{itemize}

\hypertarget{method-individual-country-wise-binary-logistic-regressions}{%
\subsection{Method: individual country-wise binary logistic
regressions}\label{method-individual-country-wise-binary-logistic-regressions}}

Now that the variables have been explained, a brief explanation of the
method used in this paper shall follow. It will contain per-country
binary logistic regressions, with results being reported both as
log-odds coefficients to determine the influence of the independent
variables quickly (Figure 1) and odds ratio coefficients to quantify the
strength of the influence (Figure 2).

\hypertarget{analysis-and-results}{%
\section{Analysis and results}\label{analysis-and-results}}

\hypertarget{uni--and-bivariate-analysis}{%
\subsection{Uni- and bivariate
analysis}\label{uni--and-bivariate-analysis}}

First of all, looking at the dependent variable individually is
advisable. For this purpose, a small frequency table, as well as a
visualization of both the 4-level and the binary version of it have been
constructed (see Appendix A). From it, we can see that there are
generally speaking slightly more people that have a positive view of
technocracy within the sample. In the more fine-grained 4-level view, we
can also see that most of them (\textasciitilde60\%) answered in the
mid-categories, not the more `extreme' ones.

Secondly, taking a look into each country highlights nation specific
attitudes (Appendix B). It becomes clear that in Greece and Germany,
people are more critical towards technocracy than in the other
countries. In Argentina, there are more favourable views, but in
general, people are divided into two nearly equally big groups. In
Brazil and Romania though, citizens with positive attitudes towards
technocracy are the majority. This is interesting, especially when
looking at both extremes: both Greece and Romania have experience with
technocratic governments, both are located in Europe --- but the
attitudes towards the concept are radically different. This can perhaps
be explained by referencing to the financial crisis and the austerity
reforms in Greece. Citizens might link this with technocrats being in
charge, thus being more critical towards it.

\newpage{}

\hypertarget{multivariate-analysis-regression-models-per-country}{%
\subsection{Multivariate analysis: regression models per
country}\label{multivariate-analysis-regression-models-per-country}}

After this uni- and bivariate approach, the next step is to take a look
at the regression results for the countries. In a first step, the
country subsets were created. Then, as explained, a binary logistic
regression model was fit to the data per country. A graphical version
can be found in the appendix (Appendix C), for the interpretation here,
this paper will concentrate on the regression tables. For the
interpretation process, this paper orientates itself by
\protect\hyperlink{ref-wenzelburger2014statistische}{Wenzelburger et
al.} (\protect\hyperlink{ref-wenzelburger2014statistische}{2014}),
especially the chapter on logistic regression analysis (see
\protect\hyperlink{ref-wenzelburger2014statistische}{Wenzelburger et
al., 2014, pp. 55--89}). Coefficients and odds ratios are rounded to two
digits. In the case of odds ratios under the threshold of 1, reciprocal
values were reported, indicating the odds of an event \textbf{not}
happening.

From there, we can see that the first variable concerning the
\emph{left-right self-placement} of citizens is not significant.
Therefore, no statistically sound and correct inferences can be drawn
with regard to technocratic attitudes. This also means that, in the
context of this paper, there is no evidence in favour of \emph{H4}
across countries. This is also the same for the \emph{corruption
perception} of citizens, meaning that also \emph{H5} cannot be accepted
under these circumstances for any case.

\emph{Political interest} on the other hand, the effect on technocratic
attitudes is significant with citizens both in Germany and in Argentina.
Taking a look at the regression tables, we can infer from Figure 1 that
the direction is as intended (with {[}1{]} symbolizing that the
coefficient is for the politically interested citizens). A politically
interested citizen is less likely to find technocracy an appealing
concept than one that is not interested in politics. This finding is in
line with \emph{H3} for the respective countries. To quantify the
results, odds ratios are used: with the variable increasing by one unit,
the odds of \textbf{not} having technocratic attitudes increase by the
factors of 1.68 (Germany) and 1.88 (Argentina) --- ceteris paribus.

Moving on to \emph{political trust}, one can find a significant effect
for Romania, which is also in the intended direction: higher values of
political trust correspond with a less technocratic attitude of
citizens. Quantifying it again by using odds ratios, one can see that
for a one-unit increase on the political trust scale, the odds of
\textbf{not} having technocratic attitudes increases by the factor of
1.11. Evidence can therefore be found in favour of \emph{H2} for the
Romanian case.

For the effect of \emph{satisfaction with democracy}, this paper
receives (at first) counterintuitive results for Germany and Greece: the
coefficient is significant, but the direction is not as intended, with
higher values on the satisfaction scale corresponding with the occurence
of technocratic attitudes. The effect is not crucial, quantifying it via
the odds ratios again, with a one unit increase of satisfaction
corresponding with an increased odds of having technocratic attitudes of
a factor of 1.08 (Germany) and 1.10 (Greece), respectively. This paper
therefore rejects \emph{H1} in the countries, where the coefficient was
significant. The effect remains somewhat puzzling; an attempt to explain
this finding can be made by taking a look at
\protect\hyperlink{ref-lavezzolo2021will}{Lavezzolo et al.}
(\protect\hyperlink{ref-lavezzolo2021will}{2021}) again, and what has
been explained under 2.2. Form this point of view, one could argue that
it seems possible that people \emph{can} indeed be satisfied with
democracy \emph{and} in favour of technocracy in some more adapted,
within-democracy form --- that being technopols.

\emph{Age} of respondents and \emph{income} are another two variables
that do not possess statistical significance in the logistic regression
model across all countries. For citizens' \emph{educational level}, a
significant coefficient can be found in the Romanian case. The effect
points in the intended direction, indicating that with higher
educational level, one can also find more technocratic attitudes in
citizens of this country (factors: 2.02 for mid-level, 2.60 for
high-level education). The reasoning behind this may be, what was
already mentioned by \protect\hyperlink{ref-chiru2022wants}{Chiru \&
Enyedi} (\protect\hyperlink{ref-chiru2022wants}{2022}): people with
higher education tend to belong to a social elite that may see
technocracy as a means to counteract radical forces, creating a save
haven for their assets (contesting the findings of stealth democracy
literature, e. g. \protect\hyperlink{ref-coffe2014education}{Coffé \&
Michels} (\protect\hyperlink{ref-coffe2014education}{2014})). Also,
these people may even consider themselves to be experts or possess a
great amount of knowledge, rendering technocracy a viable alternative
for them to make their voices more heard in politics --- an argument in
a more similar vein to
\protect\hyperlink{ref-bickerton2020technocracy}{Bickerton \& Accetti}
(\protect\hyperlink{ref-bickerton2020technocracy}{2020}). \emph{Social
capital} is also significant in two cases (Germany and Greece), but
here, the effect remains somewhat unclear. In Germany, the analysis
finds evidence in favour of \emph{H6}: people with higher social capital
are more likely to be in favour of technocratic attitudes (factor: 1.14
for having technocratic attitudes for a one-point increase on the social
capital scale). In Greece, the effect is inverted, though: people with
higher social capital are less likely to find technocracy appealing
(factor: 1.41 for \textbf{not} having technocratic attitudes for a
one-point increase on the social capital scale). With \emph{income}
having an insignificant coefficient, the effect of \emph{educational
level}, despite its significance and effect, only being significant in
one country, and the findings regarding \emph{social capital} being
somewhat unclear, it was decided to reject H6. The findings of
\protect\hyperlink{ref-chiru2022wants}{Chiru \& Enyedi}
(\protect\hyperlink{ref-chiru2022wants}{2022}) could not exactly be
recreated in this paper, indicating that further research has to be done
in this direction.

The analysis is coming to a close with a last look at the
\emph{attitudes of citizens towards science and technology} and the
novel hypothesis \emph{H7}. Here, one can find two significant
coefficients for Germany and Brazil. The direction is not as intended in
the beginning, indicating that citizens with more positive attitudes
towards science tend to view technocracy more negatively. The effect is
rather small quantitatively, indicating factors of 1.08 (Germany) and
1.06 (Brazil) for \textbf{not} having technocratic attitudes for a
one-point increase on the scale of scientific and technological
attitudes. This points, for now, to a rejection of the novel hypothesis.
The reasoning behind this is not entirely clear. One could theorize
that, within the science community, especially with regard to
reproducibility, open standards and the open source mentality are very
prominent --- thus rendering a form of government, where only experts
decide based on what they consider uncontested truth, unattractive to
people that associate with this science mentality. In the same vein,
Karl Popper's concept of falsificationism, which features prominently
within the scientific community, should also be mentioned. When experts
govern by what they consider uncontested (i. e. positivist, as already
mentioned by \protect\hyperlink{ref-bertsou2020measuring}{Bertsou \&
Caramani} (\protect\hyperlink{ref-bertsou2020measuring}{2020})) truth,
one could answer with Popper's normative approach that in reality, no
truth could or should be seen as such (see
\protect\hyperlink{ref-popper1935logik}{Popper, 2005{[}1935{]}}). A
third reasoning behind this comes from the analysis itself. Even though
the coefficients are not significant, one can witness that in some
countries (e. g. especially in Germany or Brazil), the effect of
education is inverted: people with higher education seem to view
technocracy in a bad light, compared with their lower educated peers
(this being in line again with the stealth democracy literature, see e.
g. \protect\hyperlink{ref-coffe2014education}{Coffé \& Michels}
(\protect\hyperlink{ref-coffe2014education}{2014})) --- so, expecting
those with high education to have e. g. graduated from university and
thus as having experienced the scientific world and its values from up
close, they could possibly also be seen as more critical of technocracy
because of their (related) higher opinion of science and technology,
having been a part of this ecosystem themselves. To conclude, it shall
nevertheless be said that, at this point, this is initially just
speculation and strengthens the requirement of more in-depth research,
both qualitatively and quantitatively.

\begin{figure}

{\centering \includegraphics[width=23cm,height=22cm,angle=90]{output_LO} 

}

\caption{Regression results as log-odds coefficients}\label{fig:insert graphics: LO table}
\end{figure}

\begin{figure}

{\centering \includegraphics[width=23cm,height=22cm,angle=90]{output_OR} 

}

\caption{Regression results as odds ratios}\label{fig:insert graphics: OR table}
\end{figure}

\newpage{}

\hypertarget{discussion-of-results-final-considerations-and-the-way-to-go}{%
\section{Discussion of results, final considerations and the `way to
go'}\label{discussion-of-results-final-considerations-and-the-way-to-go}}

It is clear that the novel hypothesis concerning attitudes towards
science and technology and their effect on technocratic attitudes within
citizens are tested in a rather small space here. The results can only
be considered preliminary in nature, thus making it a desideratum to
expand research, with more data becoming available in the future and
more efforts being made in the research direction of technocratic
attitudes.

Also, another question can be addressed: the operationalization of the
variable. In this paper, as already discussed, an index of two items was
built. The value of Cronbach's alpha was also reported as satisfactory.
But there are more other possibilities, e. g. to incorporate more of the
science-focused questions from the WVS into an even more fleshed out
index. Another possibility would be to design a study centered around
these scientific attitudes and values of people, where one would have to
first come up with a fitting questionnaire. In there, the questions or
statements should revolve only around the views of respondents on
science and technology, plus established context variables like
political trust and interest. As was already mentioned throughout this
paper, the field of attitudinal research, concentrating on technocratic
attitudes in particular, is still an evolving one with lots of `moving
parts.' Scholars have not come to as many decisive conclusions as is the
case with e. g. populism. Therefore, data availability is a big
desideratum, too.

For future scholarly research, another thinkable relationship concerning
technocratic attitudes within citizens shall be proposed, which was not
included in this study due to the space limitations. As sort of a
counterpart to opinions on science and technology, also religiosity is
an imaginable variable when researching technocratic attitudes. A few
preliminary steps in this direction were done in the data preparation
part of this paper.

What this paper has shown is that there is still a lot of research to be
done in the future. A few effects that have been discussed by different
scholars have proven to hold also in this particular setting, but not
all of them provided uncontested results. The reason of this is clearly
the ever-evolving nature of the topic --- which is on the other hand
also opening up a lot of new opportunities and research spaces for
political scientists to explore new focal points within the subject.

\newpage{}

\hypertarget{appendix}{%
\section{Appendix}\label{appendix}}

\hypertarget{declaration-of-authorship}{%
\subsection{Declaration of authorship}\label{declaration-of-authorship}}

I hereby declare that I have independently prepared the term paper with
the title within the framework of the course `PWM-CS-HS2: Digital
Challenges to Institutions' (winter term 2021/22), and that I have used
\textbf{no} other aids than those mentioned in the list of sources and
the bibliography section. I have marked as such all passages taken
verbatim or in spirit from the sources and literature.

\vfill

Signature \hfill Bamberg, 15.04.2022

\newpage{}

\hypertarget{online-appendix}{%
\subsection{Online appendix}\label{online-appendix}}

The \href{https://github.com/FWisniewski44/digital-challenges}{online
repository of this paper on GitHub} hosts the files and the code
necessary to replicate the results of this paper. Due to the regulations
and terms of service of the WVS, the data set cannot be provided, as
redistribution in any form is not allowed. However, one can download WVS
wave 7 from their website and adjust the corresponding lines of code
after cloning the repository.

\clearpage

\hypertarget{appendix-a-univariate-frequency-table-and-graphics-for-dependent-variable}{%
\subsection{Appendix A: Univariate frequency table and graphics for
dependent
variable}\label{appendix-a-univariate-frequency-table-and-graphics-for-dependent-variable}}

\begin{longtable}[]{@{}lccccc@{}}
\caption{Citizens' attitudes towards technocracy (frequency table, four
levels)}\tabularnewline
\toprule
& Freq & \% Valid & \% Valid Cum. & \% Total & \% Total Cum. \\
\midrule
\endfirsthead
\toprule
& Freq & \% Valid & \% Valid Cum. & \% Total & \% Total Cum. \\
\midrule
\endhead
Very good & 1249 & 20.71 & 20.71 & 18.50 & 18.50 \\
Fairly good & 2213 & 36.69 & 57.40 & 32.79 & 51.29 \\
Fairly Bad & 1456 & 24.14 & 81.55 & 21.57 & 72.86 \\
Very bad & 1113 & 18.45 & 100.00 & 16.49 & 89.35 \\
& 719 & NA & NA & 10.65 & 100.00 \\
Total & 6750 & 100.00 & 100.00 & 100.00 & 100.00 \\
\bottomrule
\end{longtable}

\begin{figure}

{\centering \includegraphics[width=15cm,height=8cm]{seminarPaper_files/figure-latex/frequencies in data: 4 level scale-1} 

}

\caption{Citizens' attitudes towards technocracy (4-level scale)}\label{fig:frequencies in data: 4 level scale}
\end{figure}

\begin{figure}

{\centering \includegraphics[width=15cm,height=8cm]{seminarPaper_files/figure-latex/frequencies in data: binary scale-1} 

}

\caption{Citizens' attitudes towards technocracy (binary)}\label{fig:frequencies in data: binary scale}
\end{figure}

\clearpage

\hypertarget{appendix-b-bivariate-analysis-of-dependent-variable-per-country}{%
\subsection{Appendix B: Bivariate analysis of dependent variable per
country}\label{appendix-b-bivariate-analysis-of-dependent-variable-per-country}}

\begin{figure}

{\centering \includegraphics[width=15cm,height=7cm]{seminarPaper_files/figure-latex/technocratic attitudes in citizens by countries --- recreation of Bertsou & Pastorella graphic with the EVS-1} 

}

\caption{Citizens' attitudes towards technocracy (standardized scale, four levels)}\label{fig:technocratic attitudes in citizens by countries --- recreation of Bertsou & Pastorella graphic with the EVS}
\end{figure}

\begin{figure}

{\centering \includegraphics[width=15cm,height=7cm]{seminarPaper_files/figure-latex/alternative plot with binary-1} 

}

\caption{Citizens' attitudes towards technocracy (standardized scale, binary)}\label{fig:alternative plot with binary}
\end{figure}

\clearpage

\hypertarget{appendix-c-regression-results-graphical-representation-of-effects}{%
\subsection{Appendix C: Regression results, graphical representation of
effects}\label{appendix-c-regression-results-graphical-representation-of-effects}}

\begin{figure}

{\centering \includegraphics[width=16cm,height=10cm]{seminarPaper_files/figure-latex/plotting of odds ratios for coefficients-1} 

}

\caption{Regression coefficients for all countries visualized}\label{fig:plotting of odds ratios for coefficients}
\end{figure}

\begin{longtable}[]{@{}ll@{}}
\toprule
\emph{Variable names} & \emph{Description} \\
\midrule
\endhead
corrPerc & Corruption Perception \\
polInt\_bin1 & Political Interest with reference category \\
educ & Educational Level \\
inc & Income \\
\bottomrule
\end{longtable}

\clearpage

\hypertarget{references}{%
\section*{References}\label{references}}
\addcontentsline{toc}{section}{References}

\hypertarget{refs}{}
\begin{CSLReferences}{1}{0}
\leavevmode\vadjust pre{\hypertarget{ref-bengtsson2009direct}{}}%
Bengtsson, Å., \& Mattila, M. (2009). Direct democracy and its critics:
Support for direct democracy and `stealth'democracy in finland.
\emph{West European Politics}, \emph{32}(5), 1031--1048.

\leavevmode\vadjust pre{\hypertarget{ref-bertsou2020measuring}{}}%
Bertsou, E., \& Caramani, D. (2020). Measuring technocracy. In \emph{The
technocratic challenge to democracy} (pp. 91--109). Routledge.
\url{http://eribertsou.com/wp-content/uploads/2021/03/Bertsou-and-Caramani-2020-Measuring-Technocracy.pdf}

\leavevmode\vadjust pre{\hypertarget{ref-bertsou2022people}{}}%
Bertsou, E., \& Caramani, D. (2022). People haven't had enough of
experts: Technocratic attitudes among citizens in nine european
democracies. \emph{American Journal of Political Science}, \emph{66}(1),
5--23.

\leavevmode\vadjust pre{\hypertarget{ref-bertsou2017technocratic}{}}%
Bertsou, E., \& Pastorella, G. (2017). Technocratic attitudes: A
citizens' perspective of expert decision-making. \emph{West European
Politics}, \emph{40}(2), 430--458.

\leavevmode\vadjust pre{\hypertarget{ref-bickerton2020technocracy}{}}%
Bickerton, C., \& Accetti, C. I. (2020). Technocracy and political
theory. In \emph{The technocratic challenge to democracy} (pp. 29--43).
Routledge.

\leavevmode\vadjust pre{\hypertarget{ref-bylund2021separation}{}}%
Bylund, P. L., \& Packard, M. D. (2021). Separation of power and
expertise: Evidence of the tyranny of experts in sweden's COVID-19
responses. \emph{Southern Economic Journal}, \emph{87}(4), 1300--1319.

\leavevmode\vadjust pre{\hypertarget{ref-caramani2020technocratic}{}}%
Caramani, D. (2020). The technocratic challenge to democracy. In E.
Bertsou \& D. Caramani (Eds.), \emph{The technocratic challenge to
democracy}. Routledge.

\leavevmode\vadjust pre{\hypertarget{ref-charron2016ideology}{}}%
Charron, N., \& Bågenholm, A. (2016). Ideology, party systems and
corruption voting in european democracies. \emph{Electoral Studies},
\emph{41}, 35--49.

\leavevmode\vadjust pre{\hypertarget{ref-chiru2022wants}{}}%
Chiru, M., \& Enyedi, Z. (2022). Who wants technocrats? A comparative
study of citizen attitudes in nine young and consolidated democracies.
\emph{The British Journal of Politics and International Relations},
\emph{24}(1), 95--112.

\leavevmode\vadjust pre{\hypertarget{ref-chiru2012voter}{}}%
Chiru, M., \& Gherghina, S. (2012). When voter loyalty fails: Party
performance and corruption in bulgaria and romania. \emph{European
Political Science Review}, \emph{4}(1), 29--49.

\leavevmode\vadjust pre{\hypertarget{ref-coffe2014education}{}}%
Coffé, H., \& Michels, A. (2014). Education and support for
representative, direct and stealth democracy. \emph{Electoral Studies},
\emph{35}, 1--11.

\leavevmode\vadjust pre{\hypertarget{ref-dahl_onDemocracy}{}}%
Dahl, R. A. (2020). \emph{On democracy}. Veritas.

\leavevmode\vadjust pre{\hypertarget{ref-donovan2006popular}{}}%
Donovan, T., \& Karp, J. A. (2006). Popular support for direct
democracy. \emph{Party Politics}, \emph{12}(5), 671--688.

\leavevmode\vadjust pre{\hypertarget{ref-ecker2016corruption}{}}%
Ecker, A., Glinitzer, K., \& Meyer, T. M. (2016). Corruption performance
voting and the electoral context. \emph{European Political Science
Review}, \emph{8}(3), 333--354.

\leavevmode\vadjust pre{\hypertarget{ref-evs2017}{}}%
EVS. (2020). \emph{European values study 2017: Integrated dataset (EVS
2017)}. GESIS Data Archive. Cologne.

\leavevmode\vadjust pre{\hypertarget{ref-font2015participation}{}}%
Font, J., Wojcieszak, M., \& Navarro, C. J. (2015). Participation,
representation and expertise: Citizen preferences for political
decision-making processes. \emph{Political Studies}, \emph{63},
153--172.

\leavevmode\vadjust pre{\hypertarget{ref-FAZ2020virologen}{}}%
Frankfurter Allgemeine Zeitung. (2020). \emph{{Diese Virologen haben das
Regiment übernommen}}. Available under:
https://www.faz.net/aktuell/gesellschaft/diese-virologen-haben-waehrend-der-corona-krise-das-regiment-uebernommen-16692489.html.

\leavevmode\vadjust pre{\hypertarget{ref-franzen2007sozialkapital}{}}%
Franzen, A., \& Pointner, S. (2007). {Sozialkapital:
Konzeptualisierungen und Messungen}. \emph{{Sozialkapital. Grundlagen
Und Anwendungen. Sonderheft}}, \emph{47}, 66--90.

\leavevmode\vadjust pre{\hypertarget{ref-ganuza2020experts}{}}%
Ganuza, E., \& Font, J. (2020). Experts in government: What for?
Ambiguities in public opinion towards technocracy. \emph{Politics and
Governance}, \emph{8}(4), 520--532.

\leavevmode\vadjust pre{\hypertarget{ref-habermas2015lure}{}}%
Habermas, J. (2015). \emph{The lure of technocracy}. John Wiley \& Sons.

\leavevmode\vadjust pre{\hypertarget{ref-wvsWave7}{}}%
Haerpfer, C. W., Inglehart, R. F., Moreno, A., Welzl, C., Kizilova, K.,
Díez-Medrano, J., Lagos, M., Norris, P., Ponarin, E., \& Puranen, B.
(2022). \emph{World values survey: Round seven --- country-pooled
datafile version 3.0}. JD Systems Institute \& WVSA Secretariat. Madrid,
Spain \& Vienna, Austria.

\leavevmode\vadjust pre{\hypertarget{ref-hawkins2018ideational}{}}%
Hawkins, K. E., \& Rovira Kaltwasser, C. (2018). Introduction. The
ideational approach. In K. E. Hawkins, R. E. Carlin, L. Littvay, \& C.
Rovira Kaltwasser (Eds.), \emph{{The Ideational Approach to Populism.
Concept, Theory, and Analysis}}. Routledge.

\leavevmode\vadjust pre{\hypertarget{ref-hibbing2002stealth}{}}%
Hibbing, J. R., \& Theiss-Morse, E. (2002). \emph{Stealth democracy:
Americans' beliefs about how government should work}. Cambridge
University Press.

\leavevmode\vadjust pre{\hypertarget{ref-hooghe2011distrusting}{}}%
Hooghe, M., Marien, S., \& Pauwels, T. (2011). {Where do distrusting
voters turn if there is no viable exit or voice option? The impact of
political trust on electoral behaviour in the Belgian regional elections
of June 2009 1}. \emph{Government and Opposition}, \emph{46}(2),
245--273.

\leavevmode\vadjust pre{\hypertarget{ref-joignant2011technopols}{}}%
Joignant, A. (2011). The politics of technopols: Resources, political
competence and collective leadership in chile, 1990-2010. \emph{Journal
of Latin American Studies}, \emph{43}(3), 517--546.

\leavevmode\vadjust pre{\hypertarget{ref-kuhlmann2021tracing}{}}%
Kuhlmann, S., Hellström, M., Ramberg, U., \& Reiter, R. (2021). Tracing
divergence in crisis governance: Responses to the COVID-19 pandemic in
france, germany and sweden compared. \emph{International Review of
Administrative Sciences}, \emph{87}(3), 556--575.

\leavevmode\vadjust pre{\hypertarget{ref-laclau2005populist}{}}%
Laclau, E. (2005). \emph{On populist reason}. Verso.

\leavevmode\vadjust pre{\hypertarget{ref-lavezzolo2021will}{}}%
Lavezzolo, S., Ramiro, L., \& Fernández-Vázquez, P. (2021). The will for
reason: Voter demand for experts in office. \emph{West European
Politics}, \emph{44}(7), 1506--1531.

\leavevmode\vadjust pre{\hypertarget{ref-mcdonnell2014defining}{}}%
McDonnell, D., \& Valbruzzi, M. (2014). Defining and classifying
technocrat-led and technocratic governments. \emph{European Journal of
Political Research}, \emph{53}(4), 654--671.

\leavevmode\vadjust pre{\hypertarget{ref-mudde2004populist}{}}%
Mudde, C. (2004). The populist zeitgeist. \emph{Government and
Opposition}, \emph{39}(4), 541--563.

\leavevmode\vadjust pre{\hypertarget{ref-muxfcller2016populismus}{}}%
Müller, J.-W. (2016). \emph{{Was ist Populismus? Ein Essay}}. Suhrkamp.

\leavevmode\vadjust pre{\hypertarget{ref-popper1935logik}{}}%
Popper, K. R. (2005{[}1935{]}). \emph{{Logik der Forschung}}. Mohr
Siebeck.

\leavevmode\vadjust pre{\hypertarget{ref-wenzelburger2014statistische}{}}%
Wenzelburger, G., Jäckle, S., \& König, P. (2014). \emph{{Weiterführende
Statistische Methoden für Politikwissenschaftler. Eine
Anwendungsbezogene Einführung mit STATA}}. De Gruyter Oldenbourg
Wissenschaftsverlag.

\leavevmode\vadjust pre{\hypertarget{ref-windholz2020governing}{}}%
Windholz, E. L. (2020). Governing in a pandemic: From parliamentary
sovereignty to autocratic technocracy. \emph{The Theory and Practice of
Legislation}, \emph{8}(1-2), 93--113.

\leavevmode\vadjust pre{\hypertarget{ref-wvs2022info}{}}%
WVS. (2022). \emph{{World Values Survey (WVS)}. Who we are}. Available
under: https://www.worldvaluessurvey.org/WVSContents.jsp.

\leavevmode\vadjust pre{\hypertarget{ref-ziller2015pure}{}}%
Ziller, C., \& Schübel, T. (2015). {'The pure people' versus 'the
corrupt elite'?} Political corruption, political trust and the success
of radical right parties in europe. \emph{Journal of Elections, Public
Opinion and Parties}, \emph{25}(3), 368--386.

\end{CSLReferences}

\end{document}
