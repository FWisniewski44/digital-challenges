% Options for packages loaded elsewhere
\PassOptionsToPackage{unicode}{hyperref}
\PassOptionsToPackage{hyphens}{url}
\PassOptionsToPackage{dvipsnames,svgnames,x11names}{xcolor}
%
\documentclass[
  12pt,
  english,
]{article}
\title{\includegraphics[width=4cm,height=\textheight]{/Users/flo/Documents/backup_doc_linux/uni-bamberg.png}

Technocracy as a challenge to representative democracy}
\usepackage{etoolbox}
\makeatletter
\providecommand{\subtitle}[1]{% add subtitle to \maketitle
  \apptocmd{\@title}{\par {\large #1 \par}}{}{}
}
\makeatother
\subtitle{Citizens' attitudes towards technocracy and what to make of
it}
\author{Florian Wisniewski\footnote{Otto-Friedrich-Universität Bamberg
  (Matriculation Nr.: 2028075) - 3rd semester, M. A. Political Science} \and Am
Werkkanal 9, 96047 Bamberg \textbar{}
\href{mailto:florian.wisniewski@stud.uni-bamberg.de}{\nolinkurl{florian.wisniewski@stud.uni-bamberg.de}}}
\date{15-04-2022}

\usepackage{amsmath,amssymb}
\usepackage{lmodern}
\usepackage{setspace}
\usepackage{iftex}
\ifPDFTeX
  \usepackage[T1]{fontenc}
  \usepackage[utf8]{inputenc}
  \usepackage{textcomp} % provide euro and other symbols
\else % if luatex or xetex
  \usepackage{unicode-math}
  \defaultfontfeatures{Scale=MatchLowercase}
  \defaultfontfeatures[\rmfamily]{Ligatures=TeX,Scale=1}
  \setmainfont[]{Times New Roman}
\fi
% Use upquote if available, for straight quotes in verbatim environments
\IfFileExists{upquote.sty}{\usepackage{upquote}}{}
\IfFileExists{microtype.sty}{% use microtype if available
  \usepackage[]{microtype}
  \UseMicrotypeSet[protrusion]{basicmath} % disable protrusion for tt fonts
}{}
\makeatletter
\@ifundefined{KOMAClassName}{% if non-KOMA class
  \IfFileExists{parskip.sty}{%
    \usepackage{parskip}
  }{% else
    \setlength{\parindent}{0pt}
    \setlength{\parskip}{6pt plus 2pt minus 1pt}}
}{% if KOMA class
  \KOMAoptions{parskip=half}}
\makeatother
\usepackage{xcolor}
\IfFileExists{xurl.sty}{\usepackage{xurl}}{} % add URL line breaks if available
\IfFileExists{bookmark.sty}{\usepackage{bookmark}}{\usepackage{hyperref}}
\hypersetup{
  pdfauthor={Florian Wisniewski; Am Werkkanal 9, 96047 Bamberg \textbar{} florian.wisniewski@stud.uni-bamberg.de},
  pdflang={EN},
  colorlinks=true,
  linkcolor={blue},
  filecolor={Maroon},
  citecolor={blue},
  urlcolor={blue},
  pdfcreator={LaTeX via pandoc}}
\urlstyle{same} % disable monospaced font for URLs
\usepackage[left=2.5cm,right=2.5cm,top=2cm,bottom=2cm]{geometry}
\usepackage{graphicx}
\makeatletter
\def\maxwidth{\ifdim\Gin@nat@width>\linewidth\linewidth\else\Gin@nat@width\fi}
\def\maxheight{\ifdim\Gin@nat@height>\textheight\textheight\else\Gin@nat@height\fi}
\makeatother
% Scale images if necessary, so that they will not overflow the page
% margins by default, and it is still possible to overwrite the defaults
% using explicit options in \includegraphics[width, height, ...]{}
\setkeys{Gin}{width=\maxwidth,height=\maxheight,keepaspectratio}
% Set default figure placement to htbp
\makeatletter
\def\fps@figure{htbp}
\makeatother
\setlength{\emergencystretch}{3em} % prevent overfull lines
\providecommand{\tightlist}{%
  \setlength{\itemsep}{0pt}\setlength{\parskip}{0pt}}
\setcounter{secnumdepth}{5}
\newlength{\cslhangindent}
\setlength{\cslhangindent}{1.5em}
\newlength{\csllabelwidth}
\setlength{\csllabelwidth}{3em}
\newlength{\cslentryspacingunit} % times entry-spacing
\setlength{\cslentryspacingunit}{\parskip}
\newenvironment{CSLReferences}[2] % #1 hanging-ident, #2 entry spacing
 {% don't indent paragraphs
  \setlength{\parindent}{0pt}
  % turn on hanging indent if param 1 is 1
  \ifodd #1
  \let\oldpar\par
  \def\par{\hangindent=\cslhangindent\oldpar}
  \fi
  % set entry spacing
  \setlength{\parskip}{#2\cslentryspacingunit}
 }%
 {}
\usepackage{calc}
\newcommand{\CSLBlock}[1]{#1\hfill\break}
\newcommand{\CSLLeftMargin}[1]{\parbox[t]{\csllabelwidth}{#1}}
\newcommand{\CSLRightInline}[1]{\parbox[t]{\linewidth - \csllabelwidth}{#1}\break}
\newcommand{\CSLIndent}[1]{\hspace{\cslhangindent}#1}
\ifXeTeX
  % Load polyglossia as late as possible: uses bidi with RTL langages (e.g. Hebrew, Arabic)
  \usepackage{polyglossia}
  \setmainlanguage[]{english}
\else
  \usepackage[main=english]{babel}
% get rid of language-specific shorthands (see #6817):
\let\LanguageShortHands\languageshorthands
\def\languageshorthands#1{}
\fi
\ifLuaTeX
  \usepackage{selnolig}  % disable illegal ligatures
\fi

\begin{document}
\maketitle
\begin{abstract}
placeholder for abstract.
\end{abstract}

\setstretch{1.5}
\newpage{}

\tableofcontents

\newpage{}

\hypertarget{introduction}{%
\section{Introduction}\label{introduction}}

One of the most discussed themes in political science over the last
years most definitely was \emph{populism}. Especially in the US -- with
the rise of Donald Trump to become the 45th president in 2017 and the
accompanying transformation of the Republican Party -- this phenomenon
evolved into a well-researched topic. But not only there: also in
Europe, the rise of populism and populist politicians became a big
issue, also within the scientific community - namely triggered by e. g.
SYRIZA's rise to power in Greece, the emergence of the AfD as a
right-wing populist party in Germany, or the presidential race in France
being decided between the liberal candidate Macron and the right-wing
populist Marine Le Pen. When talking and writing about it, one comes
across many different definitions to the same term populism: Cas Mudde's
`populism as a thin ideology'
(\protect\hyperlink{ref-mudde2004populist}{Mudde, 2004}), Ernesto
Laclau's `populism as a discursive strategy'
(\protect\hyperlink{ref-laclau2005populist}{Laclau, 2005}), Jan-Werner
Müller's `methods of populism'
(\protect\hyperlink{ref-muxfcller2016populismus}{Müller, 2016}) or the
`ideational approach' by Kirk Hawkins and Cristóbal Rovira Kaltwasser
(\protect\hyperlink{ref-hawkins2018ideational}{Hawkins \& Rovira
Kaltwasser, 2018}).

But populism is by far not the only `alternative approach' to
representative democracy. One of the other possible views is
technocracy, and research within this space is still very fluid and
evolving in this field within political science. But a lot of said
research has so far often concentrated on the more normative aspects of
technocracy (\protect\hyperlink{ref-habermas2015lure}{Habermas, 2015})
or has taken a rather expert-centered stance, discussing the
determination of who the experts would be and who chooses when someone
is considered one
(\protect\hyperlink{ref-bickerton2020technocracy}{Bickerton \& Accetti,
2020}). But there is more to \emph{`technocracy' as a whole topic} than
that

This paper seeks to take a more citizen-centered approach: the question
that will be addressed is about their attitudes towards this alternative
concept of government and governance.Other social scientists, such as
Eri Bertsou and Guilia Pastorella, have taken up this narrative before
and have shown that there is significant homogeneity throughout Europe's
people when talking about their opinions of technocracy
(\protect\hyperlink{ref-bertsou2017technocratic}{Bertsou \& Pastorella,
2017}). However, the data they used was from back in 2009; this makes it
highly likely that citizens' attitudes towards technocracy may have
evolved and changed over time, especially considering events after 2009
up to the origin of the newer data set in 2017, like the Brexit
referendum or the aftermath of the financial crisis in Europe.

Therefore, taking survey data about citizens' stances towards
technocracy in European countries, the this exploratory analysis aims at
shedding more light upon the real-world implications of technocracy as a
normative concept with updated data and see if previously obtained
results nowadays still hold. Also, as a secondary goal, it shall be
examined if there are certain patterns within the cross-national data
that promise even more interesting research perspectives for the future.

\newpage{}

\hypertarget{theoretical-considerations}{%
\section{Theoretical considerations}\label{theoretical-considerations}}

\hypertarget{what-even-is-technocracy-really}{%
\subsection{What even is technocracy,
really?}\label{what-even-is-technocracy-really}}

As mentioned in the introductory paragraph, technocracy could be
considered a still evolving field of action within the social sciences.
But that does not mean that there is nothing to draw from in theory.
Technocracy as a concept is first mentioned or described by Plato
(stemming from the Greek \emph{`techne'}, translated as `art' or
`craft'). As explained by Bickerton and Accetti, this is what can be
considered the classical argument for technocracy: Plato believed that
to really fulfill the goal of \emph{good} government, i. e. to justly
rule and bring order to the social lives of the people within society,
the one who governs needs to have suitable skills to do so. Those, they
further interpret Plato, can be found in the philosophers: they would
know about and have the skills required to do so (see
\protect\hyperlink{ref-bickerton2020technocracy}{Bickerton \& Accetti,
2020, p. 32f}.).

Another account of what can be considered technocracy has been given by
Robert A. Dahl in `On Democracy.' He takes a position of staunch
resistance from a democracy-theoretical point of view, stating:
\emph{Because experts may be qualified to serve as your agents does not
mean that they are qualified to serve as your rulers.}
(\protect\hyperlink{ref-dahl_onDemocracy}{Dahl, 2000})

Thereby, he made the exact opposite point to the argument Plato made,
that was described before. The latter assumed the philosopher-kings (so
to say, the proto-technocrats) to be the best possible rulers because
they have a certain set of skills that the broad masses do not have,
thus causing chaos when democracy is the form of governance. Dahl on the
other hand brings forth the notion of \emph{corruption through power} -
by which the pursuit of the `greater good' for the whole society, which
was being projected onto the philosopher-kings, can be undermined.

\hypertarget{the-difficult-relationship-between-technocracy-and-populism-and-technocratic-vs.-populist-attitudes}{%
\subsection{The difficult relationship between technocracy and populism
and technocratic vs.~populist
attitudes}\label{the-difficult-relationship-between-technocracy-and-populism-and-technocratic-vs.-populist-attitudes}}

\hypertarget{hypotheses}{%
\subsection{(Hypotheses?)}\label{hypotheses}}

\hypertarget{methods-and-data}{%
\section{Methods and data}\label{methods-and-data}}

Theoretically and practically building upon
\protect\hyperlink{ref-bertsou2017technocratic}{Bertsou \& Pastorella}
(\protect\hyperlink{ref-bertsou2017technocratic}{2017}), this paper also
utilizes data from the European Values Study (EVS).

\hypertarget{analysis-and-results}{%
\section{Analysis and results}\label{analysis-and-results}}

\hypertarget{discussion-of-results-and-final-considerations}{%
\section{Discussion of results and final
considerations}\label{discussion-of-results-and-final-considerations}}

\newpage{}

\hypertarget{references}{%
\section{References}\label{references}}

\hypertarget{refs}{}
\begin{CSLReferences}{1}{0}
\leavevmode\vadjust pre{\hypertarget{ref-bertsou2017technocratic}{}}%
Bertsou, E., \& Pastorella, G. (2017). Technocratic attitudes: A
citizens' perspective of expert decision-making. \emph{West European
Politics}, \emph{40}(2), 430--458.

\leavevmode\vadjust pre{\hypertarget{ref-bickerton2020technocracy}{}}%
Bickerton, C., \& Accetti, C. I. (2020). Technocracy and political
theory. In \emph{The technocratic challenge to democracy} (pp. 29--43).
Routledge.

\leavevmode\vadjust pre{\hypertarget{ref-dahl_onDemocracy}{}}%
Dahl, R. A. (2000). \emph{On democracy}. Yale University Press.

\leavevmode\vadjust pre{\hypertarget{ref-evs2017}{}}%
EVS. (2020). \emph{European values study 2017: Integrated dataset (EVS
2017)}. GESIS Data Archive, Cologne.

\leavevmode\vadjust pre{\hypertarget{ref-habermas2015lure}{}}%
Habermas, J. (2015). \emph{The lure of technocracy}. John Wiley \& Sons.

\leavevmode\vadjust pre{\hypertarget{ref-hawkins2018ideational}{}}%
Hawkins, K. E., \& Rovira Kaltwasser, C. (2018). Introduction. The
ideational approach. In K. E. Hawkins, R. E. Carlin, L. Littvay, \& C.
Rovira Kaltwasser (Eds.), \emph{The ideational approach to populism.
Concept, theory, and analysis}. Routledge.

\leavevmode\vadjust pre{\hypertarget{ref-laclau2005populist}{}}%
Laclau, E. (2005). \emph{On populist reason}. Verso.

\leavevmode\vadjust pre{\hypertarget{ref-mudde2004populist}{}}%
Mudde, C. (2004). The populist zeitgeist. \emph{Government and
Opposition}, \emph{39}(4), 541--563.

\leavevmode\vadjust pre{\hypertarget{ref-muxfcller2016populismus}{}}%
Müller, J.-W. (2016). \emph{Was ist populismus? Ein essay}. Suhrkamp.

\end{CSLReferences}

\newpage{}

\hypertarget{appendix}{%
\section{Appendix}\label{appendix}}

\end{document}
