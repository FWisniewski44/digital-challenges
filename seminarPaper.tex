% Options for packages loaded elsewhere
\PassOptionsToPackage{unicode}{hyperref}
\PassOptionsToPackage{hyphens}{url}
\PassOptionsToPackage{dvipsnames,svgnames,x11names}{xcolor}
%
\documentclass[
  12pt,
  english,
]{article}
\title{\includegraphics[width=4cm,height=\textheight]{/Users/flo/Documents/backup_doc_linux/uni-bamberg.png}

Technocracy as a challenge to representative democracy}
\usepackage{etoolbox}
\makeatletter
\providecommand{\subtitle}[1]{% add subtitle to \maketitle
  \apptocmd{\@title}{\par {\large #1 \par}}{}{}
}
\makeatother
\subtitle{Citizens' attitudes towards technocracy and what to make of
it}
\author{Florian Wisniewski\footnote{Otto-Friedrich-Universität Bamberg
  (Matriculation Nr.: 2028075) - 3rd semester, M. A. Political Science} \and Am
Werkkanal 9, 96047 Bamberg \textbar{}
\href{mailto:florian.wisniewski@stud.uni-bamberg.de}{\nolinkurl{florian.wisniewski@stud.uni-bamberg.de}}}
\date{15-04-2022}

\usepackage{amsmath,amssymb}
\usepackage{lmodern}
\usepackage{setspace}
\usepackage{iftex}
\ifPDFTeX
  \usepackage[T1]{fontenc}
  \usepackage[utf8]{inputenc}
  \usepackage{textcomp} % provide euro and other symbols
\else % if luatex or xetex
  \usepackage{unicode-math}
  \defaultfontfeatures{Scale=MatchLowercase}
  \defaultfontfeatures[\rmfamily]{Ligatures=TeX,Scale=1}
  \setmainfont[]{Tahoma}
\fi
% Use upquote if available, for straight quotes in verbatim environments
\IfFileExists{upquote.sty}{\usepackage{upquote}}{}
\IfFileExists{microtype.sty}{% use microtype if available
  \usepackage[]{microtype}
  \UseMicrotypeSet[protrusion]{basicmath} % disable protrusion for tt fonts
}{}
\makeatletter
\@ifundefined{KOMAClassName}{% if non-KOMA class
  \IfFileExists{parskip.sty}{%
    \usepackage{parskip}
  }{% else
    \setlength{\parindent}{0pt}
    \setlength{\parskip}{6pt plus 2pt minus 1pt}}
}{% if KOMA class
  \KOMAoptions{parskip=half}}
\makeatother
\usepackage{xcolor}
\IfFileExists{xurl.sty}{\usepackage{xurl}}{} % add URL line breaks if available
\IfFileExists{bookmark.sty}{\usepackage{bookmark}}{\usepackage{hyperref}}
\hypersetup{
  pdfauthor={Florian Wisniewski; Am Werkkanal 9, 96047 Bamberg \textbar{} florian.wisniewski@stud.uni-bamberg.de},
  pdflang={EN},
  colorlinks=true,
  linkcolor={blue},
  filecolor={Maroon},
  citecolor={blue},
  urlcolor={blue},
  pdfcreator={LaTeX via pandoc}}
\urlstyle{same} % disable monospaced font for URLs
\usepackage[left=2.5cm,right=2.5cm,top=2cm,bottom=2cm]{geometry}
\usepackage{graphicx}
\makeatletter
\def\maxwidth{\ifdim\Gin@nat@width>\linewidth\linewidth\else\Gin@nat@width\fi}
\def\maxheight{\ifdim\Gin@nat@height>\textheight\textheight\else\Gin@nat@height\fi}
\makeatother
% Scale images if necessary, so that they will not overflow the page
% margins by default, and it is still possible to overwrite the defaults
% using explicit options in \includegraphics[width, height, ...]{}
\setkeys{Gin}{width=\maxwidth,height=\maxheight,keepaspectratio}
% Set default figure placement to htbp
\makeatletter
\def\fps@figure{htbp}
\makeatother
\setlength{\emergencystretch}{3em} % prevent overfull lines
\providecommand{\tightlist}{%
  \setlength{\itemsep}{0pt}\setlength{\parskip}{0pt}}
\setcounter{secnumdepth}{5}
\newlength{\cslhangindent}
\setlength{\cslhangindent}{1.5em}
\newlength{\csllabelwidth}
\setlength{\csllabelwidth}{3em}
\newlength{\cslentryspacingunit} % times entry-spacing
\setlength{\cslentryspacingunit}{\parskip}
\newenvironment{CSLReferences}[2] % #1 hanging-ident, #2 entry spacing
 {% don't indent paragraphs
  \setlength{\parindent}{0pt}
  % turn on hanging indent if param 1 is 1
  \ifodd #1
  \let\oldpar\par
  \def\par{\hangindent=\cslhangindent\oldpar}
  \fi
  % set entry spacing
  \setlength{\parskip}{#2\cslentryspacingunit}
 }%
 {}
\usepackage{calc}
\newcommand{\CSLBlock}[1]{#1\hfill\break}
\newcommand{\CSLLeftMargin}[1]{\parbox[t]{\csllabelwidth}{#1}}
\newcommand{\CSLRightInline}[1]{\parbox[t]{\linewidth - \csllabelwidth}{#1}\break}
\newcommand{\CSLIndent}[1]{\hspace{\cslhangindent}#1}
\ifXeTeX
  % Load polyglossia as late as possible: uses bidi with RTL langages (e.g. Hebrew, Arabic)
  \usepackage{polyglossia}
  \setmainlanguage[]{english}
\else
  \usepackage[main=english]{babel}
% get rid of language-specific shorthands (see #6817):
\let\LanguageShortHands\languageshorthands
\def\languageshorthands#1{}
\fi
\ifLuaTeX
  \usepackage{selnolig}  % disable illegal ligatures
\fi

\begin{document}
\maketitle
\begin{abstract}
This paper will revolve around the topic of technocracy. Whilst it has
been examined in detail on state level, research about citizens'
attitudes towards this concept of government is still rather scarce.
Building remotely upon the hypotheses and findings of Bertsou \&
Pastorella (2017), this paper has a twofold goal: first, it will
function at least partially as a replication study to aforementioned
research, testing parts of the results for validity with newer data
available. Secondly, new inferences could be drawn (if found).
\end{abstract}

\setstretch{1.5}
\newpage{}

\tableofcontents

\newpage{}

\hypertarget{introduction}{%
\section{Introduction}\label{introduction}}

One of the most discussed themes in political science over the last
years most definitely was \emph{populism}. Especially in the US -- with
the rise of Donald Trump to become the 45th president in 2017 and the
accompanying transformation of the Republican Party -- this phenomenon
evolved into a well-researched topic. But not only there: also in
Europe, the rise of populism and populist politicians became a big
issue, also within the scientific community - namely triggered by e. g.
left-wing populist SYRIZA's rise to power in Greece, the emergence of
the AfD as a right-wing populist party in Germany, or the presidential
race in France being decided between the liberal candidate Macron and
the right-wing or even far right populist Marine Le Pen. When talking
and writing about it, one comes across many different definitions to the
same term populism: Cas Mudde's `populism as a thin ideology'
(\protect\hyperlink{ref-mudde2004populist}{Mudde, 2004}), Ernesto
Laclau's `populism as a discursive strategy'
(\protect\hyperlink{ref-laclau2005populist}{Laclau, 2005}), Jan-Werner
Müller's `methods of populism'
(\protect\hyperlink{ref-muxfcller2016populismus}{Müller, 2016}) or the
`ideational approach' by Kirk Hawkins and Cristóbal Rovira Kaltwasser
(\protect\hyperlink{ref-hawkins2018ideational}{Hawkins \& Rovira
Kaltwasser, 2018}).

But populism is by far not the only `alternative approach' to
representative democracy. One of the other possible views is
technocracy, and research within this space is still very fluid and
evolving in this field within political science. A lot of said research
has so far often concentrated on the more normative aspects of
technocracy (\protect\hyperlink{ref-habermas2015lure}{Habermas, 2015})
or has taken a rather expert-centered stance, discussing the
determination of who the experts would be and who chooses when someone
is considered one
(\protect\hyperlink{ref-bickerton2020technocracy}{Bickerton \& Accetti,
2020}). But there is more to \emph{`technocracy' as a whole topic} than
that. For example, one could also ask the question of how the citizens
themselves see this concept in turn.

This paper seeks to take this more citizen-centered approach: the
question that will be addressed is about their attitudes towards this
alternative concept of government and governance. Other social
scientists have very recently taken up this narrative before and have
shown that there is significant homogeneity throughout Europe's people
when talking about their opinions of technocracy (see e. g.
\protect\hyperlink{ref-bertsou2017technocratic}{Bertsou \& Pastorella}
(\protect\hyperlink{ref-bertsou2017technocratic}{2017}),
\protect\hyperlink{ref-ganuza2020experts}{Ganuza \& Font}
(\protect\hyperlink{ref-ganuza2020experts}{2020}),
\protect\hyperlink{ref-lavezzolo2021will}{Lavezzolo et al.}
(\protect\hyperlink{ref-lavezzolo2021will}{2021}),
\protect\hyperlink{ref-chiru2022wants}{Chiru \& Enyedi}
(\protect\hyperlink{ref-chiru2022wants}{2022}) or
\protect\hyperlink{ref-bertsou2022people}{Bertsou \& Caramani}
(\protect\hyperlink{ref-bertsou2022people}{2022})). However, the data
they used was from back in 2009; this makes it highly likely that
citizens' attitudes towards technocracy may have evolved and changed
over time, especially considering events after 2009 up to the origin of
the newer data set in 2017, like the Brexit referendum or the aftermath
of the financial crisis in Europe.

Therefore, taking survey data about citizens' stances towards
technocracy in European countries, the this exploratory analysis aims at
shedding more light upon the real-world implications of technocracy as a
normative concept with updated data and see if previously obtained
results nowadays still hold. Also, as a secondary goal, it shall be
examined if there are certain patterns within the cross-national data
that promise even more interesting research perspectives for the future.

\newpage{}

\hypertarget{theoretical-considerations}{%
\section{Theoretical considerations}\label{theoretical-considerations}}

\hypertarget{conceptual-background-defining-the-concept-of-technocracy-and}{%
\subsection{Conceptual background: defining the concept of technocracy
and}\label{conceptual-background-defining-the-concept-of-technocracy-and}}

As mentioned in the introductory paragraph, technocracy could be
considered a still evolving field of action within the social sciences.
But that does not mean that there is nothing to draw from in theory.
Technocracy as a concept is first mentioned or described by Plato
(stemming from the Greek \emph{`techne'}, translated as `art' or
`craft'). As explained by Bickerton and Accetti, this is what can be
considered the classical argument for technocracy: Plato believed that
to really fulfill the goal of \emph{good} government, i. e. to justly
rule and bring order to the social lives of the people within society,
the one who governs needs to have suitable skills to do so. Those, they
further interpret Plato, can be found in the philosophers: they would
know about and have the skills required to do so (see
\protect\hyperlink{ref-bickerton2020technocracy}{Bickerton \& Accetti,
2020, p. 32f}.).

Another account of what can be considered technocracy has been given by
Robert A. Dahl in his classic `On Democracy' in the chapter `Ideal
Democracy.' He takes a position of staunch resistance from a
democracy-theoretical point of view, stating: \emph{``Because experts
may be qualified to serve as your agents does not mean that they are
qualified to serve as your rulers.''}
(\protect\hyperlink{ref-dahl_onDemocracy}{Dahl, 2020, p. 73}) Thereby,
he made the exact opposite point to Plato's argument described before.
The latter assumed the philosopher-kings (so to say, the
proto-technocrats) to be the best possible rulers because they have a
certain set of skills that the broad masses do not have, thus causing
chaos when democracy is the form of governance. Dahl on the other hand
brings forth the notion of \emph{corruption through power} --- by which
the pursuit of the `greater good' for the whole society, which was being
projected onto the philosopher-kings, can be undermined.

To further shed light on the term `technocracy,' the definition given by
\protect\hyperlink{ref-caramani2020technocratic}{Caramani}
(\protect\hyperlink{ref-caramani2020technocratic}{2020}) should be
considered. He

\hypertarget{the-road-so-far-what-has-been-found-out-about-technocracy-on-state--and-individual-level}{%
\subsection{The road so far: what has been found out about technocracy
on state- and individual
level?}\label{the-road-so-far-what-has-been-found-out-about-technocracy-on-state--and-individual-level}}

Now, after having defined in detail the implications of technocratic
systems, the area of tension between representative democracy and
technocracy, and the problem this paper wants to address, it is also
necessary to take a look at what research so far has found out.

To start off, \protect\hyperlink{ref-mcdonnell2014defining}{McDonnell \&
Valbruzzi} (\protect\hyperlink{ref-mcdonnell2014defining}{2014}) point
out an important point, also relevant in the context of this paper:
there is a recognizably different thrust throughout Europe when talking
about the degree to which technocracy may be witnessed within national
governments. They set three basic rules as to when a government is to be
regarded as an ideal type technocratic one (see
\protect\hyperlink{ref-mcdonnell2014defining}{McDonnell \& Valbruzzi,
2014, p. 656}):

\begin{enumerate}
\def\labelenumi{\arabic{enumi}.}
\tightlist
\item
  Elected officials of parties do not make all governmental decisions.
\item
  Policies do not have their roots within parties which after deciding
  them, act to implement them.
\item
  High officials such as (prime) ministers do not have a party
  background.
\end{enumerate}

By applying these rules, they classify governments in Europe between
1945 and 2013. They also discern between full technocratic governments
and technocrat-led party governments. Examining their results, they see
two implications: technocratic governments form more easily in countries
where there is a stronger head of state, and this government type is
more frequently observed in countries with weaker party systems or
those, that only recently developed a democratic political system (see
\protect\hyperlink{ref-mcdonnell2014defining}{McDonnell \& Valbruzzi,
2014, p. 666}).

\protect\hyperlink{ref-font2015participation}{Font et al.}
(\protect\hyperlink{ref-font2015participation}{2015}) apply these
findings in their study about

\hypertarget{hypotheses}{%
\subsection{(Hypotheses?)}\label{hypotheses}}

\hypertarget{methods-and-data}{%
\section{Methods and data}\label{methods-and-data}}

Theoretically and practically building upon
\protect\hyperlink{ref-bertsou2017technocratic}{Bertsou \& Pastorella}
(\protect\hyperlink{ref-bertsou2017technocratic}{2017}), this paper also
utilizes data from the European Values Study (EVS).

\hypertarget{analysis-and-results}{%
\section{Analysis and results}\label{analysis-and-results}}

\hypertarget{discussion-of-results-and-final-considerations}{%
\section{Discussion of results and final
considerations}\label{discussion-of-results-and-final-considerations}}

\newpage{}

\hypertarget{references}{%
\section{References}\label{references}}

\hypertarget{refs}{}
\begin{CSLReferences}{1}{0}
\leavevmode\vadjust pre{\hypertarget{ref-bertsou2022people}{}}%
Bertsou, E., \& Caramani, D. (2022). People haven't had enough of
experts: Technocratic attitudes among citizens in nine european
democracies. \emph{American Journal of Political Science}, \emph{66}(1),
5--23.

\leavevmode\vadjust pre{\hypertarget{ref-bertsou2017technocratic}{}}%
Bertsou, E., \& Pastorella, G. (2017). Technocratic attitudes: A
citizens' perspective of expert decision-making. \emph{West European
Politics}, \emph{40}(2), 430--458.

\leavevmode\vadjust pre{\hypertarget{ref-bickerton2020technocracy}{}}%
Bickerton, C., \& Accetti, C. I. (2020). Technocracy and political
theory. In \emph{The technocratic challenge to democracy} (pp. 29--43).
Routledge.

\leavevmode\vadjust pre{\hypertarget{ref-caramani2020technocratic}{}}%
Caramani, D. (2020). The technocratic challenge to democracy. In E.
Bertsou \& D. Caramani (Eds.), \emph{The technocratic challenge to
democracy}. Routledge.

\leavevmode\vadjust pre{\hypertarget{ref-chiru2022wants}{}}%
Chiru, M., \& Enyedi, Z. (2022). Who wants technocrats? A comparative
study of citizen attitudes in nine young and consolidated democracies.
\emph{The British Journal of Politics and International Relations},
\emph{24}(1), 95--112.

\leavevmode\vadjust pre{\hypertarget{ref-dahl_onDemocracy}{}}%
Dahl, R. A. (2020). \emph{On democracy}. Veritas.

\leavevmode\vadjust pre{\hypertarget{ref-evs2017}{}}%
EVS. (2020). \emph{European values study 2017: Integrated dataset (EVS
2017)}. GESIS Data Archive, Cologne.

\leavevmode\vadjust pre{\hypertarget{ref-font2015participation}{}}%
Font, J., Wojcieszak, M., \& Navarro, C. J. (2015). Participation,
representation and expertise: Citizen preferences for political
decision-making processes. \emph{Political Studies}, \emph{63},
153--172.

\leavevmode\vadjust pre{\hypertarget{ref-ganuza2020experts}{}}%
Ganuza, E., \& Font, J. (2020). Experts in government: What for?
Ambiguities in public opinion towards technocracy. \emph{Politics and
Governance}, \emph{8}(4), 520--532.

\leavevmode\vadjust pre{\hypertarget{ref-habermas2015lure}{}}%
Habermas, J. (2015). \emph{The lure of technocracy}. John Wiley \& Sons.

\leavevmode\vadjust pre{\hypertarget{ref-hawkins2018ideational}{}}%
Hawkins, K. E., \& Rovira Kaltwasser, C. (2018). Introduction. The
ideational approach. In K. E. Hawkins, R. E. Carlin, L. Littvay, \& C.
Rovira Kaltwasser (Eds.), \emph{The ideational approach to populism.
Concept, theory, and analysis}. Routledge.

\leavevmode\vadjust pre{\hypertarget{ref-laclau2005populist}{}}%
Laclau, E. (2005). \emph{On populist reason}. Verso.

\leavevmode\vadjust pre{\hypertarget{ref-lavezzolo2021will}{}}%
Lavezzolo, S., Ramiro, L., \& Fernández-Vázquez, P. (2021). The will for
reason: Voter demand for experts in office. \emph{West European
Politics}, \emph{44}(7), 1506--1531.

\leavevmode\vadjust pre{\hypertarget{ref-mcdonnell2014defining}{}}%
McDonnell, D., \& Valbruzzi, M. (2014). Defining and classifying
technocrat-led and technocratic governments. \emph{European Journal of
Political Research}, \emph{53}(4), 654--671.

\leavevmode\vadjust pre{\hypertarget{ref-mudde2004populist}{}}%
Mudde, C. (2004). The populist zeitgeist. \emph{Government and
Opposition}, \emph{39}(4), 541--563.

\leavevmode\vadjust pre{\hypertarget{ref-muxfcller2016populismus}{}}%
Müller, J.-W. (2016). \emph{Was ist populismus? Ein essay}. Suhrkamp.

\end{CSLReferences}

\newpage{}

\hypertarget{appendix}{%
\section{Appendix}\label{appendix}}

\end{document}
