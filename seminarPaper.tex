% Options for packages loaded elsewhere
\PassOptionsToPackage{unicode}{hyperref}
\PassOptionsToPackage{hyphens}{url}
\PassOptionsToPackage{dvipsnames,svgnames,x11names}{xcolor}
%
\documentclass[
  12pt,
  english,
]{article}
\title{\includegraphics[width=4cm,height=\textheight]{/Users/flo/Documents/backup_doc_linux/uni-bamberg.png}

Technocracy as a challenge to representative democracy}
\usepackage{etoolbox}
\makeatletter
\providecommand{\subtitle}[1]{% add subtitle to \maketitle
  \apptocmd{\@title}{\par {\large #1 \par}}{}{}
}
\makeatother
\subtitle{Citizens' attitudes towards technocracy and what to make of
it}
\author{Florian Wisniewski\footnote{Otto-Friedrich-Universität Bamberg
  (Matriculation Nr.: 2028075) - 3rd semester, M. A. Political Science} \and Am
Werkkanal 9, 96047 Bamberg \textbar{}
\href{mailto:florian.wisniewski@stud.uni-bamberg.de}{\nolinkurl{florian.wisniewski@stud.uni-bamberg.de}}}
\date{15-04-2022}

\usepackage{amsmath,amssymb}
\usepackage{lmodern}
\usepackage{setspace}
\usepackage{iftex}
\ifPDFTeX
  \usepackage[T1]{fontenc}
  \usepackage[utf8]{inputenc}
  \usepackage{textcomp} % provide euro and other symbols
\else % if luatex or xetex
  \usepackage{unicode-math}
  \defaultfontfeatures{Scale=MatchLowercase}
  \defaultfontfeatures[\rmfamily]{Ligatures=TeX,Scale=1}
  \setmainfont[]{Times New Roman}
\fi
% Use upquote if available, for straight quotes in verbatim environments
\IfFileExists{upquote.sty}{\usepackage{upquote}}{}
\IfFileExists{microtype.sty}{% use microtype if available
  \usepackage[]{microtype}
  \UseMicrotypeSet[protrusion]{basicmath} % disable protrusion for tt fonts
}{}
\makeatletter
\@ifundefined{KOMAClassName}{% if non-KOMA class
  \IfFileExists{parskip.sty}{%
    \usepackage{parskip}
  }{% else
    \setlength{\parindent}{0pt}
    \setlength{\parskip}{6pt plus 2pt minus 1pt}}
}{% if KOMA class
  \KOMAoptions{parskip=half}}
\makeatother
\usepackage{xcolor}
\IfFileExists{xurl.sty}{\usepackage{xurl}}{} % add URL line breaks if available
\IfFileExists{bookmark.sty}{\usepackage{bookmark}}{\usepackage{hyperref}}
\hypersetup{
  pdfauthor={Florian Wisniewski; Am Werkkanal 9, 96047 Bamberg \textbar{} florian.wisniewski@stud.uni-bamberg.de},
  pdflang={EN},
  colorlinks=true,
  linkcolor={blue},
  filecolor={Maroon},
  citecolor={blue},
  urlcolor={blue},
  pdfcreator={LaTeX via pandoc}}
\urlstyle{same} % disable monospaced font for URLs
\usepackage[left=2.5cm,right=2.5cm,top=2cm,bottom=2cm]{geometry}
\usepackage{color}
\usepackage{fancyvrb}
\newcommand{\VerbBar}{|}
\newcommand{\VERB}{\Verb[commandchars=\\\{\}]}
\DefineVerbatimEnvironment{Highlighting}{Verbatim}{commandchars=\\\{\}}
% Add ',fontsize=\small' for more characters per line
\usepackage{framed}
\definecolor{shadecolor}{RGB}{248,248,248}
\newenvironment{Shaded}{\begin{snugshade}}{\end{snugshade}}
\newcommand{\AlertTok}[1]{\textcolor[rgb]{0.94,0.16,0.16}{#1}}
\newcommand{\AnnotationTok}[1]{\textcolor[rgb]{0.56,0.35,0.01}{\textbf{\textit{#1}}}}
\newcommand{\AttributeTok}[1]{\textcolor[rgb]{0.77,0.63,0.00}{#1}}
\newcommand{\BaseNTok}[1]{\textcolor[rgb]{0.00,0.00,0.81}{#1}}
\newcommand{\BuiltInTok}[1]{#1}
\newcommand{\CharTok}[1]{\textcolor[rgb]{0.31,0.60,0.02}{#1}}
\newcommand{\CommentTok}[1]{\textcolor[rgb]{0.56,0.35,0.01}{\textit{#1}}}
\newcommand{\CommentVarTok}[1]{\textcolor[rgb]{0.56,0.35,0.01}{\textbf{\textit{#1}}}}
\newcommand{\ConstantTok}[1]{\textcolor[rgb]{0.00,0.00,0.00}{#1}}
\newcommand{\ControlFlowTok}[1]{\textcolor[rgb]{0.13,0.29,0.53}{\textbf{#1}}}
\newcommand{\DataTypeTok}[1]{\textcolor[rgb]{0.13,0.29,0.53}{#1}}
\newcommand{\DecValTok}[1]{\textcolor[rgb]{0.00,0.00,0.81}{#1}}
\newcommand{\DocumentationTok}[1]{\textcolor[rgb]{0.56,0.35,0.01}{\textbf{\textit{#1}}}}
\newcommand{\ErrorTok}[1]{\textcolor[rgb]{0.64,0.00,0.00}{\textbf{#1}}}
\newcommand{\ExtensionTok}[1]{#1}
\newcommand{\FloatTok}[1]{\textcolor[rgb]{0.00,0.00,0.81}{#1}}
\newcommand{\FunctionTok}[1]{\textcolor[rgb]{0.00,0.00,0.00}{#1}}
\newcommand{\ImportTok}[1]{#1}
\newcommand{\InformationTok}[1]{\textcolor[rgb]{0.56,0.35,0.01}{\textbf{\textit{#1}}}}
\newcommand{\KeywordTok}[1]{\textcolor[rgb]{0.13,0.29,0.53}{\textbf{#1}}}
\newcommand{\NormalTok}[1]{#1}
\newcommand{\OperatorTok}[1]{\textcolor[rgb]{0.81,0.36,0.00}{\textbf{#1}}}
\newcommand{\OtherTok}[1]{\textcolor[rgb]{0.56,0.35,0.01}{#1}}
\newcommand{\PreprocessorTok}[1]{\textcolor[rgb]{0.56,0.35,0.01}{\textit{#1}}}
\newcommand{\RegionMarkerTok}[1]{#1}
\newcommand{\SpecialCharTok}[1]{\textcolor[rgb]{0.00,0.00,0.00}{#1}}
\newcommand{\SpecialStringTok}[1]{\textcolor[rgb]{0.31,0.60,0.02}{#1}}
\newcommand{\StringTok}[1]{\textcolor[rgb]{0.31,0.60,0.02}{#1}}
\newcommand{\VariableTok}[1]{\textcolor[rgb]{0.00,0.00,0.00}{#1}}
\newcommand{\VerbatimStringTok}[1]{\textcolor[rgb]{0.31,0.60,0.02}{#1}}
\newcommand{\WarningTok}[1]{\textcolor[rgb]{0.56,0.35,0.01}{\textbf{\textit{#1}}}}
\usepackage{graphicx}
\makeatletter
\def\maxwidth{\ifdim\Gin@nat@width>\linewidth\linewidth\else\Gin@nat@width\fi}
\def\maxheight{\ifdim\Gin@nat@height>\textheight\textheight\else\Gin@nat@height\fi}
\makeatother
% Scale images if necessary, so that they will not overflow the page
% margins by default, and it is still possible to overwrite the defaults
% using explicit options in \includegraphics[width, height, ...]{}
\setkeys{Gin}{width=\maxwidth,height=\maxheight,keepaspectratio}
% Set default figure placement to htbp
\makeatletter
\def\fps@figure{htbp}
\makeatother
\setlength{\emergencystretch}{3em} % prevent overfull lines
\providecommand{\tightlist}{%
  \setlength{\itemsep}{0pt}\setlength{\parskip}{0pt}}
\setcounter{secnumdepth}{5}
\newlength{\cslhangindent}
\setlength{\cslhangindent}{1.5em}
\newlength{\csllabelwidth}
\setlength{\csllabelwidth}{3em}
\newlength{\cslentryspacingunit} % times entry-spacing
\setlength{\cslentryspacingunit}{\parskip}
\newenvironment{CSLReferences}[2] % #1 hanging-ident, #2 entry spacing
 {% don't indent paragraphs
  \setlength{\parindent}{0pt}
  % turn on hanging indent if param 1 is 1
  \ifodd #1
  \let\oldpar\par
  \def\par{\hangindent=\cslhangindent\oldpar}
  \fi
  % set entry spacing
  \setlength{\parskip}{#2\cslentryspacingunit}
 }%
 {}
\usepackage{calc}
\newcommand{\CSLBlock}[1]{#1\hfill\break}
\newcommand{\CSLLeftMargin}[1]{\parbox[t]{\csllabelwidth}{#1}}
\newcommand{\CSLRightInline}[1]{\parbox[t]{\linewidth - \csllabelwidth}{#1}\break}
\newcommand{\CSLIndent}[1]{\hspace{\cslhangindent}#1}
\usepackage{array}
\usepackage{caption}
\usepackage{graphicx}
\usepackage{siunitx}
\usepackage[normalem]{ulem}
\usepackage{colortbl}
\usepackage{multirow}
\usepackage{hhline}
\usepackage{calc}
\usepackage{tabularx}
\usepackage{threeparttable}
\usepackage{wrapfig}
\usepackage{adjustbox}
\usepackage{hyperref}
\ifXeTeX
  % Load polyglossia as late as possible: uses bidi with RTL langages (e.g. Hebrew, Arabic)
  \usepackage{polyglossia}
  \setmainlanguage[]{english}
\else
  \usepackage[main=english]{babel}
% get rid of language-specific shorthands (see #6817):
\let\LanguageShortHands\languageshorthands
\def\languageshorthands#1{}
\fi
\ifLuaTeX
  \usepackage{selnolig}  % disable illegal ligatures
\fi

\begin{document}
\maketitle
\begin{abstract}
This paper will revolve around the topic of technocracy. Whilst it has
been examined in detail on state level, research about citizens'
attitudes towards this concept of government is still rather scarce.
Building remotely upon the hypotheses and findings of Bertsou \&
Pastorella (2017), this paper has a twofold goal: first, it will
function at least partially as a replication study to aforementioned
research, testing parts of the results for validity with newer data
available. Secondly, new inferences could be drawn (if found).
\end{abstract}

\setstretch{1.5}
\newpage{}

\tableofcontents

\newpage{}

\hypertarget{introduction}{%
\section{Introduction}\label{introduction}}

One of the most discussed themes in political science over the last
years most definitely was \emph{populism}. Especially in the US -- with
the rise of Donald Trump to become the 45th president in 2017 and the
accompanying transformation of the Republican Party -- this phenomenon
evolved into a well-researched topic. But not only there: also in
Europe, the rise of populism and populist politicians became a big
issue, also within the scientific community - namely triggered by e. g.
left-wing populist SYRIZA's rise to power in Greece, the emergence of
the AfD as a right-wing populist party in Germany, or the presidential
race in France being decided between the liberal candidate Macron and
the right-wing or even far right populist Marine Le Pen. When talking
and writing about it, one comes across many different definitions to the
same term populism: Cas Mudde's `populism as a thin ideology'
(\protect\hyperlink{ref-mudde2004populist}{Mudde, 2004}), Ernesto
Laclau's `populism as a discursive strategy'
(\protect\hyperlink{ref-laclau2005populist}{Laclau, 2005}), Jan-Werner
Müller's `methods of populism'
(\protect\hyperlink{ref-muxfcller2016populismus}{Müller, 2016}) or the
`ideational approach' by Kirk Hawkins and Cristóbal Rovira Kaltwasser
(\protect\hyperlink{ref-hawkins2018ideational}{Hawkins \& Rovira
Kaltwasser, 2018}).

But populism is by far not the only `alternative approach' to
representative democracy. One of the other possible views is
technocracy, and research within this space is still very fluid and
evolving in this field within political science. A lot of said research
has so far often concentrated on the more normative aspects of
technocracy (\protect\hyperlink{ref-habermas2015lure}{Habermas, 2015})
or has taken a rather expert-centered stance, discussing the
determination of who the experts would be and who chooses when someone
is considered one
(\protect\hyperlink{ref-bickerton2020technocracy}{Bickerton \& Accetti,
2020}). But there is more to \emph{`technocracy' as a whole topic} than
that. For example, one could also ask the question of how the citizens
themselves see this concept in turn.

This paper seeks to take this more citizen-centered approach: the
question that will be addressed is about their attitudes towards this
alternative concept of government and governance. Other social
scientists have very recently taken up this narrative before and have
shown that there is significant homogeneity throughout Europe's people
when talking about their opinions of technocracy (see e. g.
\protect\hyperlink{ref-bertsou2017technocratic}{Bertsou \& Pastorella}
(\protect\hyperlink{ref-bertsou2017technocratic}{2017}),
\protect\hyperlink{ref-ganuza2020experts}{Ganuza \& Font}
(\protect\hyperlink{ref-ganuza2020experts}{2020}),
\protect\hyperlink{ref-lavezzolo2021will}{Lavezzolo et al.}
(\protect\hyperlink{ref-lavezzolo2021will}{2021}),
\protect\hyperlink{ref-chiru2022wants}{Chiru \& Enyedi}
(\protect\hyperlink{ref-chiru2022wants}{2022}) or
\protect\hyperlink{ref-bertsou2022people}{Bertsou \& Caramani}
(\protect\hyperlink{ref-bertsou2022people}{2022})). However, the data
they used was from back in 2009; this makes it highly likely that
citizens' attitudes towards technocracy may have evolved and changed
over time, especially considering events after 2009 up to the origin of
the newer data set in 2017, like the Brexit referendum or the aftermath
of the financial crisis in Europe.

Therefore, taking survey data about citizens' stances towards
technocracy in European countries, this exploratory analysis aims at
shedding more light upon the real-world implications of technocracy as a
normative concept with updated data and see if previously obtained
results nowadays still hold. Also, as a secondary goal, it shall be
examined if there are certain patterns within the cross-national data
that promise even more interesting research perspectives for the future.

\newpage{}

\hypertarget{theoretical-considerations}{%
\section{Theoretical considerations}\label{theoretical-considerations}}

\hypertarget{conceptual-background-defining-the-concept-of-technocracy-and}{%
\subsection{Conceptual background: defining the concept of technocracy
and}\label{conceptual-background-defining-the-concept-of-technocracy-and}}

As mentioned in the introductory paragraph, technocracy could be
considered a still evolving field of action within the social sciences.
But that does not mean that there is nothing to draw from in theory.
Technocracy as a concept is first mentioned or described by Plato
(stemming from the Greek \emph{`techne'}, translated as `art' or
`craft'). As explained by Bickerton and Accetti, this is what can be
considered the classical argument for technocracy: Plato believed that
to really fulfill the goal of \emph{good} government, i. e. to justly
rule and bring order to the social lives of the people within society,
the one who governs needs to have suitable skills to do so. Those, they
further interpret Plato, can be found in the philosophers: they would
know about and have the skills required to do so (see
\protect\hyperlink{ref-bickerton2020technocracy}{Bickerton \& Accetti,
2020, p. 32f}.).

Another aspect that describes the difference between
democratic/representative and technocratic systems has been given by
Robert A. Dahl in his classic `On Democracy' in the chapter `Ideal
Democracy.' He takes a position of staunch resistance toward
technocratic expert government, stating: \emph{`Because experts may be
qualified to serve as your agents does not mean that they are qualified
to serve as your rulers.'}
(\protect\hyperlink{ref-dahl_onDemocracy}{Dahl, 2020, p. 73}) Thereby,
he made the exact opposite point to Plato's argument described before.
The latter assumed the philosopher-kings (so to say, the
`proto-technocrats') to be the best possible rulers because they have a
certain set of skills that the broad masses do not have and would apply
it to achieve the greater good for all of society --- democracy as a
form of governance, in turn, could cause chaos. Dahl on the other hand
brings forth the notion of \emph{corruption through power}, by which the
pursuit of the `greater good' for the whole society can be undermined.
\emph{Sobald Buch vorhanden, hier weiterschreiben}

To further shed light on the term `technocracy,' another definition
given by \protect\hyperlink{ref-caramani2020technocratic}{Caramani}
(\protect\hyperlink{ref-caramani2020technocratic}{2020}) should be
considered. He considers it to be `a form of power in which decisions
over the allocation of values are made by experts or technical elites
based on their knowledge, independently and in the long-term interest of
the whole of society'
(\protect\hyperlink{ref-caramani2020technocratic}{Caramani, 2020, p.
3}). He further adds four features making government technocratic in
nature (\protect\hyperlink{ref-caramani2020technocratic}{Caramani, 2020,
p. 3f}.):

\begin{enumerate}
\def\labelenumi{\arabic{enumi}.}
\item
  The source of power is knowledge, rather than popular will (e. g.
  through elections).
\item
  Political representation is realized via a „trustee model``, where
  citizens lack the knowledge and wisdom to enter the political arena
  themselves. Their „trustees`` (i. e. the experts) do that for them,
  turning democracy and elections into a mere elite selection process.
\item
  Politics is seen by the expert elites as a means to enhance the
  „greater good`` and acting in the common interest.
\item
  The common interest itself is assessed through science, rationally,
  and as objective as possible, not by ideology or derived from
  pluralist preference aggregation within the people.
\end{enumerate}

For this paper, it is also important to discuss how technocracy may be
measured. When doing attitudinal research with citizens, especially
concerning elusive and heterogenous topics like technocracy, researchers
have to be careful --- as not every respondent understands the same
under such complex concepts. To

\hypertarget{the-road-so-far-what-has-been-found-out-about-technocracy-on-state--and-individual-level}{%
\subsection{The road so far: what has been found out about technocracy
on state- and individual
level?}\label{the-road-so-far-what-has-been-found-out-about-technocracy-on-state--and-individual-level}}

Now, after having defined in detail the implications of technocratic
systems, the area of tension between representative democracy and
technocracy, and the problem this paper wants to address, it is also
necessary to take a look at what research so far has found out.

To start off, \protect\hyperlink{ref-mcdonnell2014defining}{McDonnell \&
Valbruzzi} (\protect\hyperlink{ref-mcdonnell2014defining}{2014}) point
out an important point, also relevant in the context of this paper:
there is a recognizably different thrust throughout Europe when talking
about the degree to which technocracy may be witnessed within national
governments. They set three basic rules as to when a government is to be
regarded as an ideal type technocratic one (see
\protect\hyperlink{ref-mcdonnell2014defining}{McDonnell \& Valbruzzi,
2014, p. 656}):

\begin{enumerate}
\def\labelenumi{\arabic{enumi}.}
\tightlist
\item
  Elected officials of parties do not make all governmental decisions.
\item
  Policies do not have their roots within parties which after deciding
  them, act to implement them.
\item
  High officials such as (prime) ministers do not have a party
  background.
\end{enumerate}

By applying these rules, they classify governments in Europe between
1945 and 2013. They also discern between full technocratic governments
and technocrat-led party governments. Examining their results, they see
two implications: technocratic governments form more easily in countries
where there is a stronger head of state, and this government type is
more frequently observed in countries with weaker party systems or
those, that only recently developed a democratic political system (see
\protect\hyperlink{ref-mcdonnell2014defining}{McDonnell \& Valbruzzi,
2014, p. 666}).

\protect\hyperlink{ref-font2015participation}{Font et al.}
(\protect\hyperlink{ref-font2015participation}{2015}) apply these
findings in their study about

\hypertarget{hypotheses-to-be-tested-in-this-paper}{%
\subsection{Hypotheses to be tested in this
paper}\label{hypotheses-to-be-tested-in-this-paper}}

Having explored both the theoretical background of technocracy as a
concept (including definitions of the term) under 2.1, and also some
crucial findings under 2.2, in paragraph 2.3, the hypotheses that shall
be tested in this paper will be stated. Those will consist of the
tried-and-tested ones in this research space, taken also from the
aforementioned studies under 2.2 --- but tested with newer data. Also,
to supplement scholarly work therein, two new hypotheses will be
generated and theoretically justified in the course of this paragraph.
All in all, a two step approach will be taken: both individual level and
country level hypotheses will be formulated.

\hypertarget{core-hypotheses-on-the-individual-level}{%
\subsubsection{Core hypotheses on the individual
level}\label{core-hypotheses-on-the-individual-level}}

The first of the core hypotheses having been constantly checked and
updated by scholars with newer data becoming available over the years
has been citizens' attitudes towards democracy as a form of governance
(see e. g. \protect\hyperlink{ref-bengtsson2009direct}{Bengtsson \&
Mattila} (\protect\hyperlink{ref-bengtsson2009direct}{2009}),
\protect\hyperlink{ref-coffe2014education}{Coffé \& Michels}
(\protect\hyperlink{ref-coffe2014education}{2014}) and
\protect\hyperlink{ref-bertsou2017technocratic}{Bertsou \& Pastorella}
(\protect\hyperlink{ref-bertsou2017technocratic}{2017})). Those studies
come from different directions, with some measuring citizens' support
for stealth democracy, others for technocracy and expert government.
Also, they yielded different results, as
\protect\hyperlink{ref-chiru2022wants}{Chiru \& Enyedi}
(\protect\hyperlink{ref-chiru2022wants}{2022}) point out. This is why
the hypothesis shall be tested again in the context of this paper. The
reasoning behind this is follows
\protect\hyperlink{ref-bertsou2017technocratic}{Bertsou \& Pastorella}
(\protect\hyperlink{ref-bertsou2017technocratic}{2017}): citizens that
generally trust democratic institutions and players and are satisified
with them and the system as a whole will not be fond of technocracy as a
sort-of `contender' to it. Therefore, the first hypothesis will be:

\begin{quote}
H1: The more satisfied people are with democracy as a political system,
the less likely it becomes for them to support technocracy/expert
government.
\end{quote}

The next hypothesis is grounded in the contrast of technocracy and
democracy regarding institutions. As put forth by
\protect\hyperlink{ref-bertsou2017technocratic}{Bertsou \& Pastorella}
(\protect\hyperlink{ref-bertsou2017technocratic}{2017}), if citizens
have had experiences that would diminish their trust in democracy's
representative institutions (like e. g. a series of unreliable
governments that made ineffective decisions when implementing policies),
this could lead to them being more receptive for the ideas of experts
governing --- believing in the notion that they could do it in a better
way. On the contrary, citizens that have more trust in their political
system of representative democracy would be less likely to have a
positive opinion on technocracy (see
\protect\hyperlink{ref-bertsou2017technocratic}{Bertsou \& Pastorella,
2017, p. 434f}.). Another important criticism of democratic
decision-making by technocratically minded citizens is highlighted by
\protect\hyperlink{ref-bertsou2022people}{Bertsou \& Caramani}
(\protect\hyperlink{ref-bertsou2022people}{2022}), where it is stated
that the lower trust of those people comes from their perceived
ineffectivity and irresponsibility with democratic decision-making and
-makers. This paper therefore tests the following as a second
hypothesis:

\begin{quote}
H2: Technocracy and expert government is seen more negatively when trust
in political institutions of representative democracy is higher.
\end{quote}

As already pointed out by
\protect\hyperlink{ref-bengtsson2009direct}{Bengtsson \& Mattila}
(\protect\hyperlink{ref-bengtsson2009direct}{2009}) as part of the
research on stealth democracy, it has been found out that more
right-wing oriented people favour more expert decision-making over
democratic forms of government
(\protect\hyperlink{ref-bengtsson2009direct}{Bengtsson \& Mattila, 2009,
p. 1045}). On the other end of this spatial political spectrum,
\protect\hyperlink{ref-donovan2006popular}{Donovan \& Karp}
(\protect\hyperlink{ref-donovan2006popular}{2006}) found out that with
left-wing oriented citizens, it is quite the contrary: in some
countries, those people were more likely than political `moderates' to
favour the use of popular referendums, thus they would prefer more
direct citizen intervention in politics than right-wing oriented people
(\protect\hyperlink{ref-donovan2006popular}{Donovan \& Karp, 2006, p.
681}, p.~684). Previously, this reasoning has been empirically proven by
e. g. \protect\hyperlink{ref-bertsou2017technocratic}{Bertsou \&
Pastorella} (\protect\hyperlink{ref-bertsou2017technocratic}{2017}) or
\protect\hyperlink{ref-bertsou2022people}{Bertsou \& Caramani}
(\protect\hyperlink{ref-bertsou2022people}{2022}) and also partially by
\protect\hyperlink{ref-chiru2022wants}{Chiru \& Enyedi}
(\protect\hyperlink{ref-chiru2022wants}{2022}). From that background, to
see if this will also hold in the context of this paper and to further
empirically contest the reasoning behind it, the third hypothesis reads
as follows:

\begin{quote}
H3: The further on the ideological/political right citizens perceive
themselves to be, the more likely it is for them to support
technocracy/expert government.
\end{quote}

For the fourth hypothesis, this paper turns again to the findings of
\protect\hyperlink{ref-chiru2022wants}{Chiru \& Enyedi}
(\protect\hyperlink{ref-chiru2022wants}{2022}). In their study, they
have taken the state-level hypothesis of
\protect\hyperlink{ref-bertsou2017technocratic}{Bertsou \& Pastorella}
(\protect\hyperlink{ref-bertsou2017technocratic}{2017}), but have
applied it on the citizen-level. They justified this by stating that
there were other factors that, for corruption perception, had
interaction effects on the country-level, therefore making a
citizen-level necessary. As also shown by
\protect\hyperlink{ref-chiru2012voter}{Chiru \& Gherghina}
(\protect\hyperlink{ref-chiru2012voter}{2012}) and
\protect\hyperlink{ref-ecker2016corruption}{Ecker et al.}
(\protect\hyperlink{ref-ecker2016corruption}{2016}), incumbent
governments, parties, and politicians may well be held accountable for
their corruption, if perceived by citizens --- yet again, on the other
hand, it has also been discussed that there is a possibility for them to
remain in power regardless, if there is no really viable alternative for
voters (see \protect\hyperlink{ref-charron2016ideology}{Charron \&
Bågenholm} (\protect\hyperlink{ref-charron2016ideology}{2016}) or
\protect\hyperlink{ref-hooghe2011distrusting}{Hooghe et al.}
(\protect\hyperlink{ref-hooghe2011distrusting}{2011})). If punished or
not, nevertheless, corruption and citizens perceiving it can contribute
to them losing faith in their current political system
(\protect\hyperlink{ref-ziller2015pure}{Ziller \& Schübel}
(\protect\hyperlink{ref-ziller2015pure}{2015})), which is why this paper
hypothesizes:

\begin{quote}
H4: Technocracy is seen more positively by citizens that perceive
corruption to be a major problem in the political system of their
country.
\end{quote}

The fourth hypothesis will revolve around citizens' educational level.
Starting off from the stealth democracy literature once again, we first
visit \protect\hyperlink{ref-hibbing2002stealth}{Hibbing \&
Theiss-Morse} (\protect\hyperlink{ref-hibbing2002stealth}{2002}) with
their seminal contribution in this area. Regarding citizens educational
background, they provided two alternative hypotheses: one relying on
\emph{cognitive mobilization theory} (better education meaning more
favourable views of direct citizen involvement in politics through
direct democracy) and the other on \emph{political dissatisfaction
theory} (better education meaning less favourable views of direct
citizen involvement through direct democracy)
(\protect\hyperlink{ref-hibbing2002stealth}{Hibbing \& Theiss-Morse,
2002, pp. 3--4}), finding evidence in favour of the \emph{political
dissatisfaction theory}-hypothesis
(\protect\hyperlink{ref-hibbing2002stealth}{Hibbing \& Theiss-Morse,
2002, p. 6}). On the other hand,
\protect\hyperlink{ref-chiru2022wants}{Chiru \& Enyedi}
(\protect\hyperlink{ref-chiru2022wants}{2022}) point out that the
findings in this area have also been contested and a bit contradictory
when taking a look at their more recent study. They refer to e. g.
\protect\hyperlink{ref-bengtsson2009direct}{Bengtsson \& Mattila}
(\protect\hyperlink{ref-bengtsson2009direct}{2009}) that found that
people with higher education tended to have lower support for both
direct and stealth democracy than those with lower education. They
themselves found a contrary effect: people with higher education show
favourable attitudes towards technocracy more often
(\protect\hyperlink{ref-chiru2022wants}{Chiru \& Enyedi, 2022, p. 109}).
This is also somewhat supported by
\protect\hyperlink{ref-bertsou2022people}{Bertsou \& Caramani}
(\protect\hyperlink{ref-bertsou2022people}{2022}), as they state:
`technocratic-minded citizens are expected to be highly educated, given
the emphasis placed on the superior skills of a knowledge elite and the
scientific approach to politics'
(\protect\hyperlink{ref-bertsou2022people}{Bertsou \& Caramani, 2022, p.
9}). Therefore, to further shed light on the relation of educational
level with technocratic attitudes and to also test the claims made by
\protect\hyperlink{ref-chiru2022wants}{Chiru \& Enyedi}
(\protect\hyperlink{ref-chiru2022wants}{2022}) about people with more
social capital and higher income levels favouring technocratic expert
government, the fifth hypothesis reads:

\begin{quote}
H5: Citizens that are better educated, have more social capital, and
have more income view technocracy in a more positive way.
\end{quote}

\hypertarget{new-hypotheses-with-regard-to-citizens-views-on-sciencetechnology}{%
\subsubsection{New hypotheses with regard to citizens' views on
science/technology}\label{new-hypotheses-with-regard-to-citizens-views-on-sciencetechnology}}

\begin{itemize}
\tightlist
\item
  H6: People that, generally speaking, have a higher opinion of science
  and technology tend to also view technocracy and government by experts
  in a more positive light.

  \begin{itemize}
  \tightlist
  \item
    reasoning: those people tend to view knowledge and expertise higher,
    therefore I would state that they would value expertise over `mere'
    representative party politicians when talking about implementing
    what they consider important in areas like digitization, health
    care, military technology, etc.
  \end{itemize}
\item
  H7: People that have stronger religious beliefs/are more
  religious/have more (religious) faith tend to view technocracy and
  government by experts/expertise in a more negative light.

  \begin{itemize}
  \tightlist
  \item
    reasoning: people that have stronger religious beliefs generally
    tend to view science (and therefore also scientific reasoning) more
    critically, putting an emphasis on God's intervention, therefore
    being more critical of `governance by reason' alone
  \end{itemize}
\end{itemize}

\hypertarget{methods-and-data}{%
\section{Methods and data}\label{methods-and-data}}

\hypertarget{data-and-caveats}{%
\subsection{Data and caveats}\label{data-and-caveats}}

This paper utilizes data from the World Values Survey (WVS), which
closely resembles the European Values Study (EVS), also used by e. g.
\protect\hyperlink{ref-bertsou2017technocratic}{Bertsou \& Pastorella}
(\protect\hyperlink{ref-bertsou2017technocratic}{2017}) in their seminal
study on citizens' technocratic attitudes, but with two differences that
have to be mentioned. First, it contains more countries from all over
the world, but less from Europe and the EU itself, which impacts the
cases selected for this study a bit. But it also contains, secondly, a
lot more variables and also implements data from different sources like
e. g. the World Bank's GDP per capita or the `Varieties of Democracy'
(V-Dem) project that measures democracy in all forms thinkable. So,
while variable availability is generally speaking better within the WVS,
the country selection (if one wishes to concentrate only on Europe) is
better with the EVS.

\hypertarget{case-selection-of-the-paper}{%
\subsection{Case selection of the
paper}\label{case-selection-of-the-paper}}

The countries selected for this paper include Germany, Greece, Romania,
Brazil, and Argentina. Reasons for choosing them are twofold: (1)
because of data availability and (2) following the argumentation of
\protect\hyperlink{ref-chiru2022wants}{Chiru \& Enyedi}
(\protect\hyperlink{ref-chiru2022wants}{2022}). They also chose those
countries (among others), pointing to different experiences with
democratic systems, but also different regime types with different
degrees of corruption, and different backgrounds with technocracy,
technocratic decision-making, and expert government --- therefore
providing for a high degree of variation between the cases, and by that,
a good means of testing hypotheses. Due to data availability, compared
to \protect\hyperlink{ref-chiru2022wants}{Chiru \& Enyedi}
(\protect\hyperlink{ref-chiru2022wants}{2022}), the Netherlands, the UK,
Switzerland and Hungary had to be dropped, leaving the countries
mentioned above to be researched.

\hypertarget{data-set-selection}{%
\subsection{Data set selection}\label{data-set-selection}}

It has furthermore been chosen to work with the WVS to test the novel
hypotheses H6 and H7 about citizens' attitude towards technocracy and
how they are influenced by their stance on scientific and technological
matters. The WVS is a

\begin{itemize}
\tightlist
\item
  also include reason to work with WVS and not EVS
\item
  describe both, and why they make fitting data sets
\item
  describe WVS in more detail than EVS
\end{itemize}

\hypertarget{variables-included-and-operationalization-of-concepts-with-the-wvs}{%
\subsection{Variables included and operationalization of concepts with
the
WVS}\label{variables-included-and-operationalization-of-concepts-with-the-wvs}}

\hypertarget{method-individual-country-wise-binary-logistic-regressions}{%
\subsection{Method: individual country-wise binary logistic
regressions}\label{method-individual-country-wise-binary-logistic-regressions}}

\hypertarget{analysis-and-results}{%
\section{Analysis and results}\label{analysis-and-results}}

\hypertarget{univariate-analysis-taking-a-look-at-the-variable-expert-for-technocratic-attitudes}{%
\subsection{Univariate analysis: taking a look at the variable `expert'
for technocratic
attitudes}\label{univariate-analysis-taking-a-look-at-the-variable-expert-for-technocratic-attitudes}}

\begin{verbatim}
## NULL
\end{verbatim}

\includegraphics{seminarPaper_files/figure-latex/unnamed-chunk-1-1.pdf}
\includegraphics{seminarPaper_files/figure-latex/unnamed-chunk-1-2.pdf}

\includegraphics{seminarPaper_files/figure-latex/technocratic attitudes in citizens by countries --- recreation of Bertsou \& Pastorella graphic with the EVS-1.pdf}

\includegraphics{seminarPaper_files/figure-latex/alternative plot-1.pdf}

\hypertarget{multivariate-analysis-regression-models-per-country}{%
\subsection{Multivariate analysis: regression models per
country}\label{multivariate-analysis-regression-models-per-country}}

This paper applies a similar strategy to modeling the effects of the
variables as \protect\hyperlink{ref-chiru2022wants}{Chiru \& Enyedi}
(\protect\hyperlink{ref-chiru2022wants}{2022}), as it also selected
identical cases as much as possible from the data available.

\begin{Shaded}
\begin{Highlighting}[]
\DocumentationTok{\#\#\#\#\#\#\#\#\#\#\#\#\#\#\#\#\#\#\#\#\#\#\#\#\#\#\#\#\#\#\#\#\#\#\#\# regression models (testing here before inclusion into paper)}

\CommentTok{\# modAll \textless{}{-} glm(data = wvs7Chiru, formula = expert\_bin \textasciitilde{} lrScale + polInt + polTrust +}
\CommentTok{\#               sex + age + educ + incc + socCap + sciAtt,}
\CommentTok{\#             family = "binomial")}

\NormalTok{modGER }\OtherTok{\textless{}{-}} \FunctionTok{glm}\NormalTok{(}\AttributeTok{data =}\NormalTok{ wvs7GER, }\AttributeTok{formula =}\NormalTok{ expert\_bin }\SpecialCharTok{\textasciitilde{}}\NormalTok{ lrScale }\SpecialCharTok{+}\NormalTok{ corrPerc }\SpecialCharTok{+}\NormalTok{ polInt }\SpecialCharTok{+}\NormalTok{ polTrust }\SpecialCharTok{+}
\NormalTok{                satDem }\SpecialCharTok{+}\NormalTok{ age }\SpecialCharTok{+}\NormalTok{ educ }\SpecialCharTok{+}\NormalTok{ incc }\SpecialCharTok{+}\NormalTok{ socCap }\SpecialCharTok{+}\NormalTok{ sciAtt, }\AttributeTok{family =} \StringTok{"binomial"}\NormalTok{)}

\NormalTok{modGRC }\OtherTok{\textless{}{-}} \FunctionTok{glm}\NormalTok{(}\AttributeTok{data =}\NormalTok{ wvs7GRC, }\AttributeTok{formula =}\NormalTok{ expert\_bin }\SpecialCharTok{\textasciitilde{}}\NormalTok{ lrScale }\SpecialCharTok{+}\NormalTok{ corrPerc }\SpecialCharTok{+}\NormalTok{ polInt }\SpecialCharTok{+}\NormalTok{ polTrust }\SpecialCharTok{+}
\NormalTok{                satDem }\SpecialCharTok{+}\NormalTok{ age }\SpecialCharTok{+}\NormalTok{ educ }\SpecialCharTok{+}\NormalTok{ incc }\SpecialCharTok{+}\NormalTok{ socCap }\SpecialCharTok{+}\NormalTok{ sciAtt, }\AttributeTok{family =} \StringTok{"binomial"}\NormalTok{)}

\NormalTok{modROU }\OtherTok{\textless{}{-}} \FunctionTok{glm}\NormalTok{(}\AttributeTok{data =}\NormalTok{ wvs7ROU, }\AttributeTok{formula =}\NormalTok{ expert\_bin }\SpecialCharTok{\textasciitilde{}}\NormalTok{ lrScale }\SpecialCharTok{+}\NormalTok{ corrPerc }\SpecialCharTok{+}\NormalTok{ polInt }\SpecialCharTok{+}\NormalTok{ polTrust }\SpecialCharTok{+}
\NormalTok{                satDem }\SpecialCharTok{+}\NormalTok{ age }\SpecialCharTok{+}\NormalTok{ educ }\SpecialCharTok{+}\NormalTok{ incc }\SpecialCharTok{+}\NormalTok{ socCap }\SpecialCharTok{+}\NormalTok{ sciAtt, }\AttributeTok{family =} \StringTok{"binomial"}\NormalTok{)}

\NormalTok{modBRA }\OtherTok{\textless{}{-}} \FunctionTok{glm}\NormalTok{(}\AttributeTok{data =}\NormalTok{ wvs7BRA, }\AttributeTok{formula =}\NormalTok{ expert\_bin }\SpecialCharTok{\textasciitilde{}}\NormalTok{ lrScale }\SpecialCharTok{+}\NormalTok{ corrPerc }\SpecialCharTok{+}\NormalTok{ polInt }\SpecialCharTok{+}\NormalTok{ polTrust }\SpecialCharTok{+}
\NormalTok{                satDem }\SpecialCharTok{+}\NormalTok{ age }\SpecialCharTok{+}\NormalTok{ educ }\SpecialCharTok{+}\NormalTok{ incc }\SpecialCharTok{+}\NormalTok{ socCap }\SpecialCharTok{+}\NormalTok{ sciAtt, }\AttributeTok{family =} \StringTok{"binomial"}\NormalTok{)}

\NormalTok{modARG }\OtherTok{\textless{}{-}} \FunctionTok{glm}\NormalTok{(}\AttributeTok{data =}\NormalTok{ wvs7ARG, }\AttributeTok{formula =}\NormalTok{ expert\_bin }\SpecialCharTok{\textasciitilde{}}\NormalTok{ lrScale }\SpecialCharTok{+}\NormalTok{corrPerc }\SpecialCharTok{+}\NormalTok{ polInt }\SpecialCharTok{+}\NormalTok{ polTrust }\SpecialCharTok{+}
\NormalTok{                satDem }\SpecialCharTok{+}\NormalTok{ age }\SpecialCharTok{+}\NormalTok{ educ }\SpecialCharTok{+}\NormalTok{ incc }\SpecialCharTok{+}\NormalTok{ socCap }\SpecialCharTok{+}\NormalTok{ sciAtt, }\AttributeTok{family =} \StringTok{"binomial"}\NormalTok{)}
\end{Highlighting}
\end{Shaded}

\begin{Shaded}
\begin{Highlighting}[]
\DocumentationTok{\#\# tab of models, log odds displayed (for effect direction)}
\FunctionTok{tab\_model}\NormalTok{(modGER, modGRC, modROU, modBRA, modARG, }\AttributeTok{p.style =} \StringTok{"numeric\_stars"}\NormalTok{,}
          \AttributeTok{title =} \StringTok{"Log{-}odds: Regression models for Germany, Greece, Romania, Brazil, and Argentina"}\NormalTok{,}
          \AttributeTok{dv.labels =} \FunctionTok{c}\NormalTok{(}\StringTok{"Germany"}\NormalTok{, }\StringTok{"Greece"}\NormalTok{, }\StringTok{"Romania"}\NormalTok{, }\StringTok{"Brazil"}\NormalTok{, }\StringTok{"Argentina"}\NormalTok{),}
          \AttributeTok{transform =} \ConstantTok{NULL}\NormalTok{,}
          \AttributeTok{file =} \StringTok{"output\_LO.html"}\NormalTok{)}
\end{Highlighting}
\end{Shaded}

Log-odds: Regression models for Germany, Greece, Romania, Brazil, and
Argentina

~

Germany

Greece

Romania

Brazil

Argentina

Predictors

Log-Odds

CI

p

Log-Odds

CI

p

Log-Odds

CI

p

Log-Odds

CI

p

Log-Odds

CI

p

(Intercept)

0.98

-0.03~--~2.00

0.057

2.12 **

0.62~--~3.65

0.006

-2.96 ***

-4.52~--~-1.46

\textless0.001

-0.07

-1.36~--~1.19

0.911

-0.22

-1.47~--~1.01

0.727

Self-Ascribed Position onLeft-Right-Scale

-0.03

-0.08~--~0.03

0.313

0.04

-0.01~--~0.08

0.096

-0.01

-0.06~--~0.03

0.564

0.00

-0.03~--~0.04

0.836

0.01

-0.03~--~0.06

0.584

Perception of Corruptionin Respondents'RespectiveCountry

-0.04

-0.09~--~0.01

0.140

-0.04

-0.14~--~0.04

0.327

0.05

-0.04~--~0.14

0.302

-0.05

-0.14~--~0.04

0.256

0.04

-0.04~--~0.13

0.344

Political Interest

-0.29 ***

-0.44~--~-0.14

\textless0.001

0.02

-0.16~--~0.19

0.859

0.20

0.00~--~0.40

0.052

0.01

-0.13~--~0.15

0.943

-0.24 **

-0.39~--~-0.10

0.001

Political Trust

0.06

-0.01~--~0.13

0.103

0.05

-0.04~--~0.14

0.266

-0.11 *

-0.21~--~-0.01

0.035

0.03

-0.04~--~0.10

0.432

-0.02

-0.10~--~0.07

0.721

Satisfaction withDemocracy per Respondent

0.10 ***

0.04~--~0.16

0.001

0.11 **

0.04~--~0.18

0.002

0.04

-0.03~--~0.11

0.262

-0.00

-0.07~--~0.06

0.915

0.00

-0.06~--~0.06

1.000

Respondents'Age

0.00

-0.01~--~0.01

0.991

-0.01

-0.02~--~0.00

0.118

-0.00

-0.01~--~0.01

0.828

-0.01

-0.02~--~0.00

0.111

0.00

-0.01~--~0.01

0.475

educ: educ 2

-0.05

-0.44~--~0.33

0.786

0.01

-0.41~--~0.42

0.959

0.66 **

0.20~--~1.16

0.007

-0.30

-0.63~--~0.03

0.079

0.24

-0.09~--~0.58

0.153

educ: educ 3

0.01

-0.41~--~0.42

0.973

-0.36

-0.80~--~0.08

0.108

0.93 **

0.37~--~1.50

0.001

-0.38

-0.83~--~0.06

0.093

0.24

-0.19~--~0.67

0.270

incc

-0.09

-0.26~--~0.08

0.298

-0.03

-0.24~--~0.17

0.790

0.20

-0.02~--~0.44

0.088

-0.08

-0.25~--~0.09

0.347

0.05

-0.10~--~0.20

0.538

Social Capital

0.11 *

0.00~--~0.21

0.045

-0.31 ***

-0.49~--~-0.14

\textless0.001

-0.13

-0.33~--~0.04

0.156

0.02

-0.15~--~0.19

0.775

0.10

-0.03~--~0.23

0.129

Attitudes towardsScience\&Technology

-0.08 *

-0.15~--~-0.02

0.015

-0.04

-0.14~--~0.05

0.412

-0.02

-0.09~--~0.06

0.640

-0.07 *

-0.12~--~-0.01

0.023

-0.04

-0.11~--~0.03

0.249

Observations

1345

1054

991

1319

812

R2 Tjur

0.052

0.039

0.033

0.012

0.025

\begin{itemize}
\tightlist
\item
  p\textless0.05~~~** p\textless0.01~~~*** p\textless0.001
\end{itemize}

\begin{Shaded}
\begin{Highlighting}[]
\FunctionTok{includeHTML}\NormalTok{(}\StringTok{"output\_LO.html"}\NormalTok{)}
\end{Highlighting}
\end{Shaded}

\begin{Shaded}
\begin{Highlighting}[]
\DocumentationTok{\#\# tab of models, odds ratios displayed (for effect strength)}
\FunctionTok{tab\_model}\NormalTok{(modGER, modGRC, modROU, modBRA, modARG, }\AttributeTok{p.style =} \StringTok{"numeric\_stars"}\NormalTok{,}
                     \AttributeTok{title =} \StringTok{"Odds ratios: Regression models for Germany, Greece, Romania, Brazil, and Argentina"}\NormalTok{,}
                     \AttributeTok{dv.labels =} \FunctionTok{c}\NormalTok{(}\StringTok{"Germany"}\NormalTok{, }\StringTok{"Greece"}\NormalTok{, }\StringTok{"Romania"}\NormalTok{, }\StringTok{"Brazil"}\NormalTok{, }\StringTok{"Argentina"}\NormalTok{),}
                    \AttributeTok{file =} \StringTok{"output\_OR.png"}\NormalTok{)}
\end{Highlighting}
\end{Shaded}

Odds ratios: Regression models for Germany, Greece, Romania, Brazil, and
Argentina

~

Germany

Greece

Romania

Brazil

Argentina

Predictors

Odds Ratios

CI

p

Odds Ratios

CI

p

Odds Ratios

CI

p

Odds Ratios

CI

p

Odds Ratios

CI

p

(Intercept)

2.67

0.97~--~7.36

0.057

8.35 **

1.86~--~38.59

0.006

0.05 ***

0.01~--~0.23

\textless0.001

0.93

0.26~--~3.27

0.911

0.80

0.23~--~2.75

0.727

Self-Ascribed Position onLeft-Right-Scale

0.97

0.92~--~1.03

0.313

1.04

0.99~--~1.09

0.096

0.99

0.94~--~1.03

0.564

1.00

0.97~--~1.04

0.836

1.01

0.97~--~1.06

0.584

Perception of Corruptionin Respondents'RespectiveCountry

0.96

0.91~--~1.01

0.140

0.96

0.87~--~1.04

0.327

1.05

0.96~--~1.15

0.302

0.95

0.87~--~1.04

0.256

1.04

0.96~--~1.14

0.344

Political Interest

0.75 ***

0.64~--~0.87

\textless0.001

1.02

0.85~--~1.21

0.859

1.22

1.00~--~1.49

0.052

1.01

0.88~--~1.16

0.943

0.79 **

0.68~--~0.91

0.001

Political Trust

1.06

0.99~--~1.14

0.103

1.05

0.96~--~1.15

0.266

0.90 *

0.81~--~0.99

0.035

1.03

0.96~--~1.10

0.432

0.98

0.90~--~1.07

0.721

Satisfaction withDemocracy per Respondent

1.11 ***

1.04~--~1.17

0.001

1.12 **

1.04~--~1.20

0.002

1.04

0.97~--~1.12

0.262

1.00

0.93~--~1.06

0.915

1.00

0.94~--~1.06

1.000

Respondents'Age

1.00

0.99~--~1.01

0.991

0.99

0.98~--~1.00

0.118

1.00

0.99~--~1.01

0.828

0.99

0.98~--~1.00

0.111

1.00

0.99~--~1.01

0.475

educ: educ 2

0.95

0.64~--~1.39

0.786

1.01

0.67~--~1.53

0.959

1.94 **

1.22~--~3.20

0.007

0.74

0.53~--~1.03

0.079

1.28

0.91~--~1.78

0.153

educ: educ 3

1.01

0.67~--~1.52

0.973

0.70

0.45~--~1.08

0.108

2.53 **

1.45~--~4.48

0.001

0.69

0.44~--~1.06

0.093

1.27

0.83~--~1.95

0.270

incc

0.91

0.77~--~1.08

0.298

0.97

0.79~--~1.19

0.790

1.22

0.98~--~1.55

0.088

0.92

0.78~--~1.10

0.347

1.05

0.90~--~1.22

0.538

Social Capital

1.11 *

1.00~--~1.24

0.045

0.73 ***

0.62~--~0.87

\textless0.001

0.87

0.72~--~1.04

0.156

1.02

0.86~--~1.21

0.775

1.11

0.97~--~1.26

0.129

Attitudes towardsScience\&Technology

0.92 *

0.86~--~0.98

0.015

0.96

0.87~--~1.06

0.412

0.98

0.91~--~1.06

0.640

0.94 *

0.89~--~0.99

0.023

0.96

0.89~--~1.03

0.249

Observations

1345

1054

991

1319

812

R2 Tjur

0.052

0.039

0.033

0.012

0.025

\begin{itemize}
\tightlist
\item
  p\textless0.05~~~** p\textless0.01~~~*** p\textless0.001
\end{itemize}

\begin{Shaded}
\begin{Highlighting}[]
\FunctionTok{includeHTML}\NormalTok{(}\StringTok{"output\_OR.png"}\NormalTok{)}
\end{Highlighting}
\end{Shaded}

\includegraphics{seminarPaper_files/figure-latex/plotting of odds ratios for coefficients-1.pdf}

\hypertarget{discussion-of-results-and-final-considerations}{%
\section{Discussion of results and final
considerations}\label{discussion-of-results-and-final-considerations}}

It is clear that those novel hypotheses are tested in a rather small
space here, thus making it a desideratum to expand it as soon as more
data may be available in the future.

\newpage{}

\hypertarget{references}{%
\section{References}\label{references}}

\hypertarget{refs}{}
\begin{CSLReferences}{1}{0}
\leavevmode\vadjust pre{\hypertarget{ref-bengtsson2009direct}{}}%
Bengtsson, Å., \& Mattila, M. (2009). Direct democracy and its critics:
Support for direct democracy and `stealth'democracy in finland.
\emph{West European Politics}, \emph{32}(5), 1031--1048.

\leavevmode\vadjust pre{\hypertarget{ref-bertsou2022people}{}}%
Bertsou, E., \& Caramani, D. (2022). People haven't had enough of
experts: Technocratic attitudes among citizens in nine european
democracies. \emph{American Journal of Political Science}, \emph{66}(1),
5--23.

\leavevmode\vadjust pre{\hypertarget{ref-bertsou2017technocratic}{}}%
Bertsou, E., \& Pastorella, G. (2017). Technocratic attitudes: A
citizens' perspective of expert decision-making. \emph{West European
Politics}, \emph{40}(2), 430--458.

\leavevmode\vadjust pre{\hypertarget{ref-bickerton2020technocracy}{}}%
Bickerton, C., \& Accetti, C. I. (2020). Technocracy and political
theory. In \emph{The technocratic challenge to democracy} (pp. 29--43).
Routledge.

\leavevmode\vadjust pre{\hypertarget{ref-caramani2020technocratic}{}}%
Caramani, D. (2020). The technocratic challenge to democracy. In E.
Bertsou \& D. Caramani (Eds.), \emph{The technocratic challenge to
democracy}. Routledge.

\leavevmode\vadjust pre{\hypertarget{ref-charron2016ideology}{}}%
Charron, N., \& Bågenholm, A. (2016). Ideology, party systems and
corruption voting in european democracies. \emph{Electoral Studies},
\emph{41}, 35--49.

\leavevmode\vadjust pre{\hypertarget{ref-chiru2022wants}{}}%
Chiru, M., \& Enyedi, Z. (2022). Who wants technocrats? A comparative
study of citizen attitudes in nine young and consolidated democracies.
\emph{The British Journal of Politics and International Relations},
\emph{24}(1), 95--112.

\leavevmode\vadjust pre{\hypertarget{ref-chiru2012voter}{}}%
Chiru, M., \& Gherghina, S. (2012). When voter loyalty fails: Party
performance and corruption in bulgaria and romania. \emph{European
Political Science Review}, \emph{4}(1), 29--49.

\leavevmode\vadjust pre{\hypertarget{ref-coffe2014education}{}}%
Coffé, H., \& Michels, A. (2014). Education and support for
representative, direct and stealth democracy. \emph{Electoral Studies},
\emph{35}, 1--11.

\leavevmode\vadjust pre{\hypertarget{ref-dahl_onDemocracy}{}}%
Dahl, R. A. (2020). \emph{On democracy}. Veritas.

\leavevmode\vadjust pre{\hypertarget{ref-donovan2006popular}{}}%
Donovan, T., \& Karp, J. A. (2006). Popular support for direct
democracy. \emph{Party Politics}, \emph{12}(5), 671--688.

\leavevmode\vadjust pre{\hypertarget{ref-ecker2016corruption}{}}%
Ecker, A., Glinitzer, K., \& Meyer, T. M. (2016). Corruption performance
voting and the electoral context. \emph{European Political Science
Review}, \emph{8}(3), 333--354.

\leavevmode\vadjust pre{\hypertarget{ref-evs2017}{}}%
EVS. (2020). \emph{European values study 2017: Integrated dataset (EVS
2017)}. GESIS Data Archive. Cologne.

\leavevmode\vadjust pre{\hypertarget{ref-font2015participation}{}}%
Font, J., Wojcieszak, M., \& Navarro, C. J. (2015). Participation,
representation and expertise: Citizen preferences for political
decision-making processes. \emph{Political Studies}, \emph{63},
153--172.

\leavevmode\vadjust pre{\hypertarget{ref-ganuza2020experts}{}}%
Ganuza, E., \& Font, J. (2020). Experts in government: What for?
Ambiguities in public opinion towards technocracy. \emph{Politics and
Governance}, \emph{8}(4), 520--532.

\leavevmode\vadjust pre{\hypertarget{ref-habermas2015lure}{}}%
Habermas, J. (2015). \emph{The lure of technocracy}. John Wiley \& Sons.

\leavevmode\vadjust pre{\hypertarget{ref-wvsWave7}{}}%
Haerpfer, C. W., Inglehart, R. F., Moreno, A., Welzl, C., Kizilova, K.,
Díez-Medrano, J., Lagos, M., Norris, P., Ponarin, E., \& Puranen, B.
(2022). \emph{World values survey: Round seven --- country-pooled
datafile version 3.0}. JD Systems Institute \& WVSA Secretariat. Madrid,
Spain \& Vienna, Austria.

\leavevmode\vadjust pre{\hypertarget{ref-hawkins2018ideational}{}}%
Hawkins, K. E., \& Rovira Kaltwasser, C. (2018). Introduction. The
ideational approach. In K. E. Hawkins, R. E. Carlin, L. Littvay, \& C.
Rovira Kaltwasser (Eds.), \emph{The ideational approach to populism.
Concept, theory, and analysis}. Routledge.

\leavevmode\vadjust pre{\hypertarget{ref-hibbing2002stealth}{}}%
Hibbing, J. R., \& Theiss-Morse, E. (2002). \emph{Stealth democracy:
Americans' beliefs about how government should work}. Cambridge
University Press.

\leavevmode\vadjust pre{\hypertarget{ref-hooghe2011distrusting}{}}%
Hooghe, M., Marien, S., \& Pauwels, T. (2011). Where do distrusting
voters turn if there is no viable exit or voice option? The impact of
political trust on electoral behaviour in the belgian regional elections
of june 2009 1. \emph{Government and Opposition}, \emph{46}(2),
245--273.

\leavevmode\vadjust pre{\hypertarget{ref-laclau2005populist}{}}%
Laclau, E. (2005). \emph{On populist reason}. Verso.

\leavevmode\vadjust pre{\hypertarget{ref-lavezzolo2021will}{}}%
Lavezzolo, S., Ramiro, L., \& Fernández-Vázquez, P. (2021). The will for
reason: Voter demand for experts in office. \emph{West European
Politics}, \emph{44}(7), 1506--1531.

\leavevmode\vadjust pre{\hypertarget{ref-mcdonnell2014defining}{}}%
McDonnell, D., \& Valbruzzi, M. (2014). Defining and classifying
technocrat-led and technocratic governments. \emph{European Journal of
Political Research}, \emph{53}(4), 654--671.

\leavevmode\vadjust pre{\hypertarget{ref-mudde2004populist}{}}%
Mudde, C. (2004). The populist zeitgeist. \emph{Government and
Opposition}, \emph{39}(4), 541--563.

\leavevmode\vadjust pre{\hypertarget{ref-muxfcller2016populismus}{}}%
Müller, J.-W. (2016). \emph{Was ist {Populismus}? Ein {Essay}}.
Suhrkamp.

\leavevmode\vadjust pre{\hypertarget{ref-ziller2015pure}{}}%
Ziller, C., \& Schübel, T. (2015). {``The pure people''} versus {``the
corrupt elite?''} Political corruption, political trust and the success
of radical right parties in europe. \emph{Journal of Elections, Public
Opinion and Parties}, \emph{25}(3), 368--386.

\end{CSLReferences}

\newpage{}

\hypertarget{appendix}{%
\section{Appendix}\label{appendix}}

\end{document}
